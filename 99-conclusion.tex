\unnumberedchapter{Záver}
V tejto práci sme uviedli problematiku návrhu a analýzy algoritmov a dátových štruktúr v pamäťových modeloch akými sú \aware a \obliv modely, v ktorých na rozdiel od klasického \RAM modelu nepovažujeme každý prístup k dátam za operáciu vykonateľnú v konštantnom čase. Prínos týchto modelov sme podložili prístupovými dobami k rôznym pamätiam v reálnych systémoch, ktoré sa líšia o niekoľko rádov.

Ďalej sme sa zamerali na \obliv model, ktorý je podstatne novší a menej známi ako \aware model. Jeho hlavným prínosom je automatická prenositeľnosť medzi systémami s rôznymi parametrami pamäťovej hierarchie a jednoduchšia implementácia. Uviedli sme niekoľko \obliv algoritmov a dátových štruktúr, ktoré sú rovnako alebo porovnateľne efektívne ako ich \aware ekvivalenty.

Hlavným výsledkom práce sú však vizualizácie vybraných \obliv dátových štruktúr, ktoré boli implementované ako rozšírenie vizualizačného programu \emph{Gnarley trees}. Tie môžu slúžiť ako učebný nástroj pre zjednodušenie a urýchlenie porozumenia týmto štruktúram.

V budúcnosti by sme chceli tieto vizualizácie rozšíriť o ďalšie \obliv dátové štruktúry, ako napríklad prioritné fronty, stromy s textovými kľúčmi, spájané zoznamy a tiež o \obliv triediace algoritmy.