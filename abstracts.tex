\chapter*{Abstrakt}

V tejto práci sa zaoberáme modelom externej pamäte, ktorý vystihuje hierarchické pamäte v dnešných výpočtových systémoch. Ďalej uvádzame upravenú verziu tohto modelu, takzvaný \emph{cache-oblivious model}. Pre tento model uvádzame niekoľko algoritmov a dátových štruktúr spolu s ich pamäťovou analýzou. Napokon v poslednej časti práce popisujeme funkcionalitu a používanie vizualizácií vytvorených ako súčasť tejto práce.

Výsledkom práce sú vizualizácie pre niekoľko vybraných \obliv dátových štruktúr, ktoré sú implementované ako rozšírenie aplikácie \emph{Gnarley Trees}, ktorá vznikla ako súčasť bakalárskej práce Jakuba Kováča v roku 2007. Na priloženom CD sa nachádza zdrojový kód a spustiteľná verzia tejto aplikácie.


\vspace{2cm}
\noindent\textsc{Kľúčové slová:} dátové štruktúry, pamäťová analýza, cache, cache-oblivious, vizualizácia

\newpage
\chapter*{Abstract}
In this thesis we consider the external memory model, which depicts the hierarchical memory used in today's computer systems. Next we introduce a modification of this model, the so called \emph{cache-oblivious model}. We describe several algorithms and data structures in this model together with their memory analyses. In the last part we describe the features and usage of visualizations created as a part of this work.

The result of this thesis are visualizations of several selected \obliv data structures implemented as an extension of the \emph{Gnarley Trees} application, which was created as a part of bachelor's thesis by Jakub Kováč in 2007. The attached CD contains source code and executable version of this application. 

\vspace{2cm}
\noindent\textsc{Keywords:} data structures, memory analyis, cache, cache-oblivious, visualizations