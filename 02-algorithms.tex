\chapter{Cache-oblivius algoritmy a dátové štruktúry}

\todo[inline]{Intro text - obsah kapitoly (alg/ds, analyza, ...)}

\section{Základné algoritmy}

Na demonštráciu \obliv algoritmov a ich analýzy v \extmem modeli použijeme jednoduchý algoritmus, ktorý počíta agregačnú funkciu nad hodnotami uloženými v poli.

\subsection{Popis algoritmu}

Majme pole $A$ veľkosti $|A|=N$ a označme jeho prvky $A = \{a_1,\dotsc,a_N\} \in X^N$. Chceme vypočítať hodnotu $f_g(A)$, kde $g: X \times Y \rightarrow Y$ je agregačná funkcia, $g_0 \in Y$ je počiatočná hodnota a $f_g: X^\infty \rightarrow Y$ je rozšírenie agregačnej funkcie definované následovne: \todo{Spravna notacia pre x infinity?}
\[
\begin{aligned}
f_g(\{a_1,\dotsc,a_k\}) &= g(a_k, f(\{a_1,\dotsc,a_{k-1}\})) \\
f_g(\emptyset) &= g_0
\end{aligned}
\]
% f(1, 2, 3) = g(3, f(1, 2)) = g(3, g(2, f(1))) = g(3, g(2, g(1, g0)))

Túto funkciu je možné implementovať jednoducho ako jeden cyklus. Schematickú verziu implementácie uvádzame v algoritme \ref{alg:aggregate}.

\begin{algorithm}
    \caption{Implementácia agregačnej funkcie $f_g$}
    \label{alg:aggregate}
    \begin{algorithmic}[1]
        \Function{aggregate}{$g, g_0, A$}
            \AlgLet{$y$}{$g_0$}
            \For{$i \gets 1,\dotsc,|A|$}
                \AlgLet{$y$}{$g(A[i], y)$}
            \EndFor
            \State \Return{$y$}
        \EndFunction
    \end{algorithmic}
\end{algorithm}

Tento algoritmus s použitím vhodnej funkcie $g$ a hodnoty $g_0$ je možné použiť na rôzne, často užitočné výpočty, ako napríklad maximum, minimum, suma a podobne:

\[
\begin{aligned}
g^{\textrm{max}}(x, y) &= \max(x, y) ~~ &g^{\textrm{max}}_0 &= -\infty \\
g^{\textrm{sum}}(x,y) &= x+y &g^{\textrm{sum}}_0 &= 0
\end{aligned}
\]


\subsection{Analýza zložitosti}
\subsubsection{Časová analýza}
Klasická časová analýza tohto algoritmu je triviálna ak uvažujeme \RAM. Keďže prístup ku každému prvku \inlcode{A[i]} zaberie konštantný čas a za predpokladu, že čas na výpočet funkcie $g$, $T_g$, je nezávislý na vstupe, bude výsledný čas na výpočet tejto funkcie

\[
T(N) = \bigO{1} + N[\bigO{1} + T_g + \bigO{1}] = T_g\cdot\bigO{N}
\]

\subsubsection{Pamäťová analýza}

V prípade \aware algoritmu by sme pole $A$ mali uložené v $\lceil \frac{N}{B} \rceil$ blokoch veľkosti $B$. Pri výpočte by sme postupne tieto bloky načítali do cache a pracovali s nimi. V rámci jedného bloku počas výpočtu nedochádza k pamäťovým presunom. Zároveň stačí každý prvok spracovať raz a teda celkový počet pamäťových operácií bude presne rovný počtu blokov, $\lceil \frac{N}{B} \rceil$. Tento algoritmus však požaduje znalosť parametra $B$ a explicitný presun blokov.

Jednoducho však vieme dosiahnuť (takmer) rovnakú zložitosť aj v prípade \obliv algoritmu \ref{alg:aggregate}, ktorý žiadne parametre pamäte zjavne nevyužíva a nepozná. Budeme predpokladať, že pole $A$ je uložené v súvislom úseku pamäte - to je možné dosiahnuť aj bez znalosti parametrov pamäte. Zvyšok algoritmu prebieha rovnako ako v predchádzajúcom prípade. Každý blok obsahujúci nejaký prvok poľa $A$ bude teda presunutý do cache práve raz, a žiadne iné presuny nenastanú. Ostáva zistiť, koľko takých blokov môže byť.

Keďže nepoznáme veľkosti blokov v pamäti, nevieme pri ukladaní prvkov poľa zaručiť zarovnanie so začiatkom bloku. V najhoršom prípade uložíme do prvého bloku iba jeden prvok. Potom bude nasledovať $\lfloor \frac{N}{B} \rfloor$ plných blokov a nakoniec ešte najviac jeden blok, ktorý opäť nie je plný. Spolu máme teda $\lfloor \frac{N}{B} \rfloor + 2$ blokov.

Pokiaľ $\lfloor \frac{N}{B} \rfloor < \lceil \frac{N}{B} \rceil$ máme spolu najviac $\lceil \frac{N}{B} \rceil + 1$ blokov. V opačnom prípade $B$ delí $N$, teda v prvom a poslednom bloku je spolu presne $B$ prvkov a medzi nimi sa nachádza najviac $\frac{N-B}{B} = \frac{N}{B} - 1$ plných. Teda blokov je vždy najviac $\lceil \frac{N}{B} \rceil +1$.

Zostrojili sme teda \obliv algoritmus s asymptoticky rovnakou zložitosťou $\bigO{\frac{N}{B}}$ ako optimálny \aware algoritmus, ktorého implementácia je však jednoduchšia, keďže nemusí explicitne spravovať presun blokov do cache.

\section{Vyhľadávacie stromy}

\todo[inline]{intro ...}
\todo[inline]{lowerbound}

\subsection{\Aware riešenie} \label{sec:static-aware}
V prípade, že poznáme veľkosť blokov $B$ v cache, môžeme problém vyhľadávacích stromov riešiť B-stromom s vetvením $\Theta(B)$. Každý vrchol teda vieme načítať s použitím $\bigO{1}$ pamäťových presunov. Výška takého B-stromu, ktorý má $N$ listov, bude $\bigO{\log_B{N}}$. Celkovo teda vyhľadávanie v tomto strome vykoná $\bigO{\log_B{N}}$ pamäťových presunov. To zodpovedá dolnej hranici vo vete \todo{dokaz}.

\subsection{Naivné \obliv riešenie}
Predtým ako popíšeme efektívne \obliv riešenie, pozrime sa na klasický binárny vyhľadávací strom. Jednoduchý a častý spôsob ako usporiadať uzly binárneho stromu v pamäti je nasledovný. Koreň uložíme na pozíciu $1$. Ľavého a pravého potomka vrchola na pozícii $x$ uložíme na pozície $2x$ a $2x+1$. Otec vrcholu $x$ bude na pozícii $\lfloor\frac{x}{2}\rfloor$. Príklad takto uloženého stromu je na obrázku \ref{fig:node_order_naive}.

Výhodou tohto usporiadania sú implicitné vzťahy medzi vrcholmi. Na udržiavanie stromu stačí jednorozmerné pole kľúčov. Na prechod medzi nimi môžeme použiť triviálne funkcie uvedene v algoritme \ref{alg:naive_indexing}.

% TODO caption/labels
% TODO side by side
\begin{algorithm}[H]
    \label{alg:naive_indexing}
	\caption{Algoritmy pre získanie pozícií ľavého syna, pravého syna a rodiča vrcholu na pozícii $x$}
	\begin{minipage}[t]{0.3\textwidth}
	    \begin{algorithmic}[1]
	        \Function{left}{$x$}
	            \State \Return{$2x$}
	        \EndFunction
	    \end{algorithmic}
	\end{minipage} \hfill
	\begin{minipage}[t]{0.3\textwidth}
	    \begin{algorithmic}[1]
	        \Function{right}{$x$}
	            \State \Return{$2x+1$}
	        \EndFunction
	    \end{algorithmic}
	\end{minipage} \hfill
	\begin{minipage}[t]{0.3\textwidth}
	    \begin{algorithmic}[1]
	        \Function{parent}{$x$}
	            \State \Return{$\lfloor\frac{x}{2}\rfloor$}
	        \EndFunction
	    \end{algorithmic}
	\end{minipage}
\end{algorithm}

Nevýhodou je však vysoký počet pamäťových presunov pri vyhľadávaní. Výška tohto stromu je $\bigO{\log{N}}$. Pri načítaní vrcholu na pozícii $x$ sa v rovnakom bloku nachádzajú vrcholy na pozíciach
\[
x-k,\dotsc,x-1,x,x+1,\dotsc,x+l
\]
kde $k+l<B$. Pri ďalšom kroku vyhľadávania budeme potrebovať vrchol $2x$ alebo $2x+1$ a teda nás pozície menšie ako $x$ nezaujímajú. V najlepšom prípade teda bude $k=0$ a $l=B-1$. Aby sa v tomto intervale nachádzali požadované vrcholy musí platiť
\[
\begin{aligned}
2x+1 &\le x+l = x+B-1 \\
x &\le B-2
\end{aligned}
\]
To znamená, že pre pozície $x > B - 2$ už bude potrebný pamäťový presun pre každý vrchol. Vrchol s pozíciou $B-2$ bude mať hĺbku $\bigO{\log{B}}$ a teda počet vrcholov na ceste z koreňa do listu, ktorých pozície v pamäti sú väčšie ako $B-2$ bude $\Omega(\log{N}-\log{B})$. Pre každý z nich je potrebné vykonať pamäťový presun a teda vyhľadávanie v takto usporiadanom binárnom strome vykoná $\Omega(\log{\frac{N}{B}})$ pamäťových presunov, čo je horšie ako pri \aware B-strome. \todo{porovnat fnc?}

\subsection{Statický \obliv vyhľadávací strom} \label{sec:static-obliv}
Problémom predošlého riešenia je neefektívne usporiadanie v pamäti - pri prístupe ku vrcholu sa spolu s ním v rovnakom bloku nachádzajú vrcholy, ktoré nie sú pre ďalší priebeh algoritmu podstatné. 

Riešením je takzvané \emph{\vEB usporiadanie} (\emph{\vEB layout}, nazvané podla \vEB stromov s podobnou myšlienkou), ktoré funguje následovne. Uvažujme úplný binárny strom výšky $h$. Ak $h=1$ tak máme iba jeden vrchol $v$ a výstupom usporiadania bude $(v)$.

Pre $h>1$ rozdelíme vstupný strom na podstrom $\tau_0$ výšky $\frac{h}{2}$, ktorého koreňom je koreň pôvodného stromu. Zostanú nám podstromy $\tau_1,\dotsc,\tau_k$, ktorých korene sú potomkovia listov $\tau_0$ a listy sú listy vstupného stromu. Rekurzívne ich uložíme do \vEB usporiadania a následne uložíme za seba, výstupom teda bude $(\tau_0,\tau_1,\dotsc,\tau_k)$. Schéma tohto delenia je na obrázku \ref{fig:veblayout_scheme} a príklad takto usporiadaného stromu na obrázku \ref{fig:node_order_veb}.

Tieto podstromy majú veľkosť $\Theta(\sqrt{N})$ kde $N$ je veľkosť vstupného stromu, keďže ich výška je $\frac{h}{2} = \frac{1}{2}\lg{N} = \lg{\sqrt{N}}$.

\begin{figure}
    \centering
    \resizebox{0.4\textwidth}{!}{
        \begin{tikzpicture}

% Triangle
\newcommand{\width}{8}
\newcommand{\height}{4}
\newcommand{\hoff}{0.25}
\newcommand{\voff}{0.25}
% Memory layout box
\newcommand{\memoff}{0.5}
\newcommand{\memw}{1}
\newcommand{\memh}{1}


\newcommand{\trianglecenter}[4]{
  \coordinate (#1) at ($1/3*(#2)+1/3*(#3)+1/3*(#4)$);
}

% Big Left/Right/Top
\coordinate (BL) at (0, 0);
\coordinate (BR) at (\width, 0);
\coordinate (BT) at (\width/2, \height);
% Top Left/Right/Top
\coordinate (TL) at (0.25*\width + \hoff, 0.5*\height);
\coordinate (TR) at (0.75*\width - \hoff, 0.5*\height);
\coordinate (TT) at ($ (BT)+(0, -\voff) $);
% Left Left/Right Right
\coordinate (LL) at ($ (\voff, \voff) + (\hoff, 0) $);
\coordinate (RL) at ($ (\width - \voff, \voff) - (\hoff, 0) $);
% Left+Right Center
\coordinate (LRC) at (0.5*\width, \voff);
\coordinate (LR) at ($ (LRC) - (0.5*\hoff, 0) $);
\coordinate (RR) at ($ (LRC) + (0.5*\hoff, 0) $);
% Centers
\trianglecenter{BC}{BL}{BR}{BT}
\trianglecenter{TC}{TL}{TR}{TT}
\trianglecenter{LC}{LL}{LR}{TL}
\trianglecenter{RC}{RL}{RR}{TR}

% Outer triangle
\draw [thick] (BL) -- (BR) -- (BT) -- (BL);
% Inner triangles
\draw (TL) -- (TR) -- (TT) -- (TL);
\draw (LL) -- (TL) -- (LR) -- (LL);
\draw (RL) -- (TR) -- (RR) -- (RL);

\node at (BC) {\Huge $\cdots$};
\node at (TC) {\Huge $\tau_0$};
\node at (LC) {\Huge $\tau_1$};
\node at (RC) {\Huge $\tau_k$};

% Layout in memory
\coordinate (M0LT) at ($ (BL) - (0, \memoff) $);
\coordinate (M0RB) at ($ (M0LT) + (\memw, -\memh) $);
\coordinate (M1RT) at ($ (M0RB) + (\memw, \memh) $);
\coordinate (M3RT) at ($ (BR) - (0, \memoff) $);
\coordinate (M3LB) at ($ (M3RT) - (\memw, \memh) $);

\draw (M0LT) rectangle (M0RB);
\draw (M0RB) rectangle (M1RT);
\draw (M1RT) rectangle (M3LB);
\draw (M3LB) rectangle (M3RT);

\node at ($ (M0LT)!0.5!(M0RB) $) {\huge $\tau_0$};
\node at ($ (M0RB)!0.5!(M1RT) $) {\huge $\tau_1$};
\node at ($ (M1RT)!0.5!(M3LB) $) {\huge $\dots$};
\node at ($ (M3LB)!0.5!(M3RT) $) {\huge $\tau_k$};


\end{tikzpicture}
    }
    \caption{Schematické znázornenie rekurzívneho delenia pri \vEB usporiadaní. Podstromy $\tau_0,\dotsc,\tau_k$ sa uložia do súvislého pola.}
    \label{fig:veblayout_scheme}
\end{figure}

\begin{figure}
    \centering
    \subbottom[Klasické usporiadanie] {
        \resizebox{0.8\textwidth}{!}{
            \begin{tikzpicture}[sibling distance = 15pt, level distance = 40pt]
\tikzset{every node/.style={circle, ultra thick, draw, minimum size=2.1em}, edge from parent/.append style={ultra thick}, edge from parent path={(\tikzparentnode) -- (\tikzchildnode)}}

% node_order.py, naive, height 5
\Tree [.\node (T) {1}; 
  [.\node (TL) {2}; 
    [.\node (TLL) {4}; 
      [.\node (TLLL) {8}; 
        [.\node (TLLLL) {16}; ]
        [.\node (TLLLR) {17}; ]]
      [.\node (TLLR) {9}; 
        [.\node (TLLRL) {18}; ]
        [.\node (TLLRR) {19}; ]]]
    [.\node (TLR) {5}; 
      [.\node (TLRL) {10}; 
        [.\node (TLRLL) {20}; ]
        [.\node (TLRLR) {21}; ]]
      [.\node (TLRR) {11}; 
        [.\node (TLRRL) {22}; ]
        [.\node (TLRRR) {23}; ]]]]
  [.\node (TR) {3}; 
    [.\node (TRL) {6}; 
      [.\node (TRLL) {12}; 
        [.\node (TRLLL) {24}; ]
        [.\node (TRLLR) {25}; ]]
      [.\node (TRLR) {13}; 
        [.\node (TRLRL) {26}; ]
        [.\node (TRLRR) {27}; ]]]
    [.\node (TRR) {7}; 
      [.\node (TRRL) {14}; 
        [.\node (TRRLL) {28}; ]
        [.\node (TRRLR) {29}; ]]
      [.\node (TRRR) {15}; 
        [.\node (TRRRL) {30}; ]
        [.\node (TRRRR) {31}; ]]]]]

% Normalize bounding box
\node [draw=none,rectangle, fit=(T) (TLLLL) (TRRRR)] (C) {};
\clip ($(C.south west) - (1,0)$) rectangle ($(C.north east) + (1, 0)$);
        
\draw [darkgray, thick,darkgray, thick, ->, shorten >= 3pt, shorten <= 3pt] (T.east) .. controls ($(T) + (5, -1)$) and ($(TL) - (5, -1)$) .. (TL.west);
\draw [darkgray, thick,darkgray, thick,->, shorten >= 3pt, shorten <= 3pt] (TL) to (TR);
\draw [darkgray, thick,->, shorten >= 3pt, shorten <= 3pt] (TR.east) .. controls ($(TR) + (5, -1.5)$) and ($(TLL) - (5, -1.5)$) .. (TLL.west);
\draw [darkgray, thick,->, shorten >= 3pt, shorten <= 3pt] (TLL) to (TLR);
\draw [darkgray, thick,->, shorten >= 3pt, shorten <= 3pt] (TLR) to (TRL);
\draw [darkgray, thick,->, shorten >= 3pt, shorten <= 3pt] (TRL) to (TRR);
\draw [darkgray, thick,->, shorten >= 3pt, shorten <= 3pt] (TRR.east) .. controls ($(TRR) + (5, -2)$) and ($(TLLL) - (5, -2)$) .. (TLLL.west);
\draw [darkgray, thick,->, shorten >= 3pt, shorten <= 3pt] (TLLL) to (TLLR);
\draw [darkgray, thick,->, shorten >= 3pt, shorten <= 3pt] (TLLR) to (TLRL);
\draw [darkgray, thick,->, shorten >= 3pt, shorten <= 3pt] (TLRL) to (TLRR);
\draw [darkgray, thick,->, shorten >= 3pt, shorten <= 3pt] (TLRR) to (TRLL);
\draw [darkgray, thick,->, shorten >= 3pt, shorten <= 3pt] (TRLL) to (TRLR);
\draw [darkgray, thick,->, shorten >= 3pt, shorten <= 3pt] (TRLR) to (TRRL);
\draw [darkgray, thick,->, shorten >= 3pt, shorten <= 3pt] (TRRL) to (TRRR);
\draw [darkgray, thick,->, shorten >= 3pt, shorten <= 3pt] (TRRR.east) .. controls ($(TRRR) + (5, -2)$) and ($(TLLLL) - (5, -2)$) .. (TLLLL.west);
\draw [darkgray, thick,->, shorten >= 3pt, shorten <= 3pt] (TLLLL) to (TLLLR);
\draw [darkgray, thick,->, shorten >= 3pt, shorten <= 3pt] (TLLLR) to (TLLRL);
\draw [darkgray, thick,->, shorten >= 3pt, shorten <= 3pt] (TLLRL) to (TLLRR);
\draw [darkgray, thick,->, shorten >= 3pt, shorten <= 3pt] (TLLRR) to (TLRLL);
\draw [darkgray, thick,->, shorten >= 3pt, shorten <= 3pt] (TLRLL) to (TLRLR);
\draw [darkgray, thick,->, shorten >= 3pt, shorten <= 3pt] (TLRLR) to (TLRRL);
\draw [darkgray, thick,->, shorten >= 3pt, shorten <= 3pt] (TLRRL) to (TLRRR);
\draw [darkgray, thick,->, shorten >= 3pt, shorten <= 3pt] (TLRRR) to (TRLLL);
\draw [darkgray, thick,->, shorten >= 3pt, shorten <= 3pt] (TRLLL) to (TRLLR);
\draw [darkgray, thick,->, shorten >= 3pt, shorten <= 3pt] (TRLLR) to (TRLRL);
\draw [darkgray, thick,->, shorten >= 3pt, shorten <= 3pt] (TRLRL) to (TRLRR);
\draw [darkgray, thick,->, shorten >= 3pt, shorten <= 3pt] (TRLRR) to (TRRLL);
\draw [darkgray, thick,->, shorten >= 3pt, shorten <= 3pt] (TRRLL) to (TRRLR);
\draw [darkgray, thick,->, shorten >= 3pt, shorten <= 3pt] (TRRLR) to (TRRRL);
\draw [darkgray, thick,->, shorten >= 3pt, shorten <= 3pt] (TRRRL) to (TRRRR);

\end{tikzpicture}
        }
        \label{fig:node_order_naive}
    }
    \subbottom[\vEB usporiadanie] {
        \resizebox{0.8\textwidth}{!}{
            \begin{tikzpicture}[sibling distance = 15pt, level distance = 40pt]
\tikzset{every node/.style={circle, ultra thick, draw, minimum size=2.1em}, edge from parent/.append style={ultra thick}, edge from parent path={(\tikzparentnode) -- (\tikzchildnode)}}
% node_order.py, van Emde Boas, height 5
\Tree [.\node (T) {1}; 
  [.\node (TL) {2}; 
    [.\node (TLL) {3}; 
      [.\node (TLLL) {5}; 
        [.\node (TLLLL) {6}; ]
        [.\node (TLLLR) {7}; ]]
      [.\node (TLLR) {8}; 
        [.\node (TLLRL) {9}; ]
        [.\node (TLLRR) {10}; ]]]
    [.\node (TLR) {4}; 
      [.\node (TLRL) {11}; 
        [.\node (TLRLL) {12}; ]
        [.\node (TLRLR) {13}; ]]
      [.\node (TLRR) {14}; 
        [.\node (TLRRL) {15}; ]
        [.\node (TLRRR) {16}; ]]]]
  [.\node (TR) {17}; 
    [.\node (TRL) {18}; 
      [.\node (TRLL) {20}; 
        [.\node (TRLLL) {21}; ]
        [.\node (TRLLR) {22}; ]]
      [.\node (TRLR) {23}; 
        [.\node (TRLRL) {24}; ]
        [.\node (TRLRR) {25}; ]]]
    [.\node (TRR) {19}; 
      [.\node (TRRL) {26}; 
        [.\node (TRRLL) {27}; ]
        [.\node (TRRLR) {28}; ]]
      [.\node (TRRR) {29}; 
        [.\node (TRRRL) {30}; ]
        [.\node (TRRRR) {31}; ]]]]]

% Normalize bounding box
\node [draw=none,rectangle, fit=(T) (TLLLL) (TRRRR)] (C) {};
\clip ($(C.south west) + (-1,-0.5)$) rectangle ($(C.north east) + (1, 0.5)$);

\node[draw, rectangle, thick, fit = (T)(TLLLL)(TRRRR), inner sep = 10] {};
\node[draw, rectangle, thin, fit = (T), inner sep = 6] {};
\node[draw, rectangle, thick, fit = (TL)(TLLLL)(TLRRR), inner sep = 6] {};
\node[draw, rectangle, thin, fit = (TL)(TLL)(TLR), inner sep = 2] {};
\node[draw, rectangle, thin, fit = (TLLL)(TLLLL)(TLLLR), inner sep = 2] {};
\node[draw, rectangle, thin, fit = (TLLR)(TLLRL)(TLLRR), inner sep = 2] {};
\node[draw, rectangle, thin, fit = (TLRL)(TLRLL)(TLRLR), inner sep = 2] {};
\node[draw, rectangle, thin, fit = (TLRR)(TLRRL)(TLRRR), inner sep = 2] {};
\node[draw, rectangle, thick, fit = (TR)(TRLLL)(TRRRR), inner sep = 6] {};
\node[draw, rectangle, thin, fit = (TR)(TRL)(TRR), inner sep = 2] {};
\node[draw, rectangle, thin, fit = (TRLL)(TRLLL)(TRLLR), inner sep = 2] {};
\node[draw, rectangle, thin, fit = (TRLR)(TRLRL)(TRLRR), inner sep = 2] {};
\node[draw, rectangle, thin, fit = (TRRL)(TRRLL)(TRRLR), inner sep = 2] {};
\node[draw, rectangle, thin, fit = (TRRR)(TRRRL)(TRRRR), inner sep = 2] {};

\end{tikzpicture}
        }
        \label{fig:node_order_veb}
    }
    \caption{Porovnanie klasického a \vEB usporiadania na úplnom binárnom strome výšky $5$. Čísla vo vrcholoch určujú poradie v pamäti.}
    \label{fig:node_order_comparison}
\end{figure}

\subsubsection{Vyhľadávanie} \label{sec:static-search}

\todo[inline]{pointers vs implicit indexing}

Pri analýze vyhľadávania sa pozrime na také podstromy predošlého delenia, že ich veľkosť je $\Theta(B)$. Ďalšie delenie a preusporiadanie je už zbytočné, no to \obliv algoritmus nemá ako vedieť. Keďže ale po rekurzívnom volaní získame len iné usporiadanie, ktoré uložíme v súvislom úseku pamäte, bude stále možné tento podstrom načítať v $\bigO{1}$ blokoch.

Majme teda vyhľadávací strom zložený z takýchto podstromov, ktorých veľkosť je medzi $\Omega(\sqrt{B})$ a $\bigO{B}$. Ich výška je teda $\Omega(\log{B})$. Pri strome výšky $\bigO{\log N}$ teda prejdeme cez $\bigO{\frac{\log{N}}{\log{B}}}$ takých podstromov a každý vyžaduje konštantný počet pamäťových presunov a spolu sa ich teda vykoná $\bigO{\log_B{N}}$, čo zodpovedá spodnej hranici tohto problému.

Máme teda vyhľadávanie, ktoré je rovnako ako pri \aware B-stromoch optimálne. Problémom tejto dátovej štruktúry je však nemožnosť efektívne vkladať či odoberať prvky - pri každej zmene by bolo potrebné strom preusporiadať. Máme teda \obliv ekvivalent statických \aware B-stromov. 

Na úpravu tohto statického stromu tak, aby efektívne zvládal operácie pridávania a odoberania, budeme potrebovať pomocnú dátovú štruktúru, ktorú popíšeme v nasledovnej sekcii.

\section{Usporiadané pole}

\todo[inline]{obrazky}
Problémom \emph{údržby usporiadaného poľa} (z anglického \emph{ordered-file maintenance}) budeme volať problém spočívajúci v udržiavaní zoradenej postupnosti prvkov, do ktorej možno pridávať nové prvky medzi ľubovolné dva existujúce a tiež prvky odstraňovať. Dátovou štruktúrou, ktorá tento problém rieši efektívne je \emph{štruktúra zhustenej pamäte} (\emph{packed-memory structure}). Táto štruktúra udržiava prvky v súvislom poli veľkosti $\bigO{N}$ s \emph{medzerami} medzi prvkami veľkosti $\bigO{1}$. Vďaka tomu bude načítanie $K$ po sebe idúcich prvkov vyžadovať $\bigO{\frac{K}{B}}$ pamäťových presunov.

\subsection{Popis štruktúry} \label{sec:of-desc}

Celá dátová štruktúra pozostáva z jedného poľa veľkosti $T = 2^k$. To (pomyselne) rozdelíme na \emph{bloky} veľkosti $S = 2^l$ tak, že $S=\Theta(\log{N})$. Počet blokov tak bude tiež mocnina dvoch.

Nad týmito blokmi zostrojíme (imaginárny) úplný binárny strom, ktorého listy sú bloky udržiavaného poľa. \emph{Hĺbkou} vrchola označíme jeho vzdialenosť od koreňa, pričom koreň má hĺbku $0$ a listy majú hĺbku $d = k-l$.

\subsection{Definície}

\emph{Kapacitou} vrchola $v$, $c(v)$, označíme počet položiek (aj prázdnych, teda aj s medzerami) poľa patriacich do blokov v podstrome začínajúcom v tomto vrchole. Kapacita listov bude teda $S$, ich rodičov $2S$ a kapacita koreňa bude $T$. Podobne budeme počet neprázdnych položiek v podstrome vrcholu $v$ volať \emph{obsadnosť} a značiť $o(v)$. 

Ďale \emph{hustotou}, $0 \le d(v) \le 1$, označíme $d(v) = \frac{o(v)}{c(v)}$. Zvoľme ľubovolné konštanty
\[
0 < \rho_d < \rho_0 < \tau_o < \tau_d < 1
\]
a definujme pre vrchol s hĺbkou $k$ \emph{dolnú} a \emph{hornú hranicu hustoty} $\rho_k$ a $\tau_k$ tak, že dostaneme postupnosť hraníc pre všetky hĺbky, pričom platí $(\rho_i,\tau_i) \subset (\rho_{i+1},\tau_{i+1})$ a teda sa tieto intervaly smerom od listov ku koreňu zmenšujú:
\[
\rho_k = \rho_0 + \frac{k}{d}(\rho_d-\rho_0) ~~
\tau_k = \tau_0 - \frac{k}{d}(\tau_0-\tau_d)
\]
\[
0 < \rho_d < \rho_{d-1} < \cdots < \rho_0 < \tau_0 < \tau_1 < \cdots < \tau_d < 1
\]

Napokon, vrchol $v$ hĺbky $k$ je \emph{v hraniciach} hustoty ak platí $\rho_k \le d(v) \le \tau_k$.

\subsection{Operácie}

\todo[inline]{pseudocode?}

\subsubsection{Vkladanie}

Implementácia operácie vkladania sa skladá z niekoľkých krokov. Najskôr zistíme, do ktorého bloku $v$ spadá pozícia, na ktorú vkladáme. Pozrieme sa, či je tento blok v hraniciach hustoty. Ak áno tak platí $d(v) < 1$ a teda $o(v) < c(v)$, čiže v tomto bloku je voľné miesto. Môžeme teda zapísať novú hodnotu do tohto bloku, pričom môže byť potrebné hodnoty v bloku popresúvať, avšak zmení sa najviac $S$ pozícii.

V opačnom prípade je tento blok mimo hraníc hustoty. Budeme postupovať hore v strome dovtedy, kým nenájdeme vrchol v hraniciach. Keďže strom je iba pomyselný, budeme túto operáciu realizovať pomocou dvoch súčasných prechodov k okrajom poľa. Počas tohto prechodu si udržiavame počet neprázdnych a všetkých pozícii a zastavíme v momente, keď hustota dosiahne požadované hranice.

Po nájdení takéhoto vrchola v hraniciach rovnomerne rozdelíme všetky hodnoty v blokoch prislúchajúcich danému podstromu. Keďže intervaly pre hranice sa smerom ku listom iba rozširujú budú po tomto popresúvaní všetky vrcholy tohto podstromu v hraniciach hustoty a teda aj požadovaný blok bude obsahovať aspoň jednu prázdnu pozíciu. Môžeme teda novú hodnotu vložiť ako v prvom kroku.

Ak nenájdeme taký vrchol, ktorého hustota by bola v hraniciach, a teda aj koreň je mimo hraníc, je táto štruktúra príliš plná. V takom prípade zostrojíme nové pole dvojnásobnej veľkosti a všetky existujúce položky, s pridanou novou, rovnomerne rozmiestnime do nového poľa.

\subsubsection{Odstraňovanie}

Operácia odstraňovania prebieha analogicky. Ako prvé požadovanú položku odstránime z prislúchajúceho bloku. Ak je tento blok aj naďalej v hraniciach hustoty tak skončíme, inak postupujeme nahor v strome, kým nenájdeme vrchol v hraniciach. Následne rovnomerne prerozdelíme položky blokoch daného podstromu.

Pokiaľ taký vrchol nenájdeme, je pole príliš prázdne a zostrojíme nové polovičnej veľkosti a rovnomerne do neho rozmiestnime zostávajúce položky pôvodného.

\subsection{Analýza}

Pri vložení aj odstraňovaní sa upraví súvislý interval $I$, ktorý sa skladá z niekoľkých blokov. Nech pri nejakej operácii došlo k prerozdeleniu prvkov v blokoch prislúchajúcich podstromu vrchola $u$. Teda pred týmto prerozdelením bol vrchol $u$ v hĺbke $k$ v hraniciach hustoty ($\rho_k \le d(u) \le \tau_k$) ale nejaký jeho potomok $v$ nebol.

Po prerozdelení budú všetky vrcholy v danom podstrome v hraniciach hustoty, avšak nie len v svojich ale aj v hraniciach pre hĺbku $k$, ktoré sú tesnejšie. Bude teda platiť $\rho_k \le d(v) \le \tau_k$. Najmenší počet operácii vloženia, $q$, potrebný na to, aby bol vrchol $v$ opäť mimo hraníc je
\[
\begin{aligned}
\frac{o(v)}{c(v)} = d(v) &\le \tau_k \hspace{3cm} & \frac{o(v)+q}{c(v)} &> \tau_{k+1} \\
o(v) &\le \tau_k c(v) & o(v)+q &> \tau_{k+1}c(v)
\end{aligned}
\]\[
q > (\tau_{k+1}-\tau_k)c(v)
\]

Podobne pre prekročenie dolnej hranice je potrebných aspoň $(\rho_k-\rho_{k+1})c(v)$ operácii odstránenia.

Pri úprave intervalu blokov v podstrome vrcholu $u$ je potrebné upraviť najviac $c(u)$ položiek, avšak táto situácia nastane pokiaľ sa potomok $v$ ocitne mimo hraníc hustoty. Priemerná veľkosť intervalu, ktorý treba preusporiadať pri vložení do podintervalu prislúchajúceho vrcholu $v$ teda bude
\[
\frac{c(u)}{(\tau_{k+1}-\tau_k)c(v)} = \frac{2c(v)}{(\tau_{k+1}-\tau_k)c(v)} = \frac{2}{\tau_{k+1}-\tau_k} = \frac{2d}{\tau_d-\tau_0} = \bigO{\log{T}}
\]
keďže $\tau_d$ a $\tau_0$ sú konštanty a výška stromu $d$ s $T$ listami je $d = \Theta(\log{T})$. Podobným spôsobom dostaneme rovnaký odhad pre odstraňovanie.

Pri vkladaní a odstraňovaní prvku ovplyvníme najviac $d$ podintervalov - tie, ktoré prislúchajú vrcholom na ceste z daného listu (bloku) do koreňa. Spolu teda bude veľkosť upraveného intervalu v priemernom prípade $\bigO{\log^2{T}}$.

\todo[inline]{worstcase bounds? conjenctured lowerbound?}
\todo[inline]{amortizacia double/half rebuildu}

Táto dátová štruktúra teda udržiava $N$ usporiadaných prvkov v poli veľkosti $\bigO{N}$ a podporuje operácie vkladania a odstraňovania, ktoré upravujú súvislý interval priemernej veľkosti $\bigO{\log^2{N}}$ a je ich teda možné realizovať pomocou $\bigO{\frac{\log^2{N}}{B}}$ pamäťových presunov.

\section{Dynamický B-strom}
\todo[inline]{nazov? nie moc bstrom ked b=2... lepsie dictionary/tree?}
\todo[inline]{intro, historia, ...}

\subsection{\aware riešenie}
V prípade \aware modelu použijeme opäť B-strom s vetvením $Theta(B)$ ako v časti \ref{sec:static-aware}. Rovnako ako pre vyhľadávanie bude na vkladanie potrebných $\bigO{\log_B{N}}$ pamäťových presunov. \todo{odvodenie?}

\subsection{Popis \obliv štruktúry}
Táto dátová štruktúra vznikne zložením predchádzajúcich dvoch - statického stromu (\ref{sec:static-obliv}) a usporiadaného poľa (\ref{sec:of-desc}) - jednoduchým spôsobom. Majme usporiadané pole veľkosti $\Theta(N)$. Keďže počet položiek v usporiadanom poli je mocnina dvoch, môžeme nad ním vybudovať statický vyhľadávací strom uložený vo \vEB usporiadaní. Listy tohto stromu budú bijektívne spárované s položkami v usporiadanom poli.

Kľúče, ktoré táto štruktúra obsahuje, sú uložené v usporiadanom poli (s medzerami). Listy statického stromu obsahujú v svojich kľúčoch rovnakú hodnotu ako k ním prislúchajúce položky usporiadaného pola. Medzeru reprezentujeme hodnotou, ktorá je pri použitom usporiadaní najmenšia. Ostatné vrcholy stromu obsahujú ako kľúč maximum z kľúčov svojich synov.

\subsection{Vyhľadávanie} \label{sec:dynamic-obliv-search}
Vyhľadávanie v tejto štruktúre prebieha jednoducho. Začínajúc od koreňa, porovnáme hľadaný kľúč s kľúčom v ľavom synovi. Pokiaľ je hľadaný väčší, bude v pravom podstrome (keďže maximum z celého ľavého podstromu je práve kľúč ľavého syna), ktorý rekurzívne prehľadáme. V opačnom prípade prehľadáme ľavý podstrom. Keď dosiahneme list, porovnáme jeho kľúč s hľadaným. Ak sa zhodujú tak sme požadovanú položku našli, inak sa v štruktúre nenachádza. \todo{dokaz ze tam nie je}

\subsubsection{Analýza}

Toto prehľadávanie prechádza cestu od koreňa k listu v strome hĺbky $\bigO{\log{N}}$ uloženom vo \vEB usporiadaní a teda rovnako ako v časti \ref{sec:static-search} vykoná $\bigO{\log_B{N}}$ pamäťových presunov.

\subsection{Vkladanie}
Zaujímavejšou operáciou je vkladanie, ktoré v pôvodnom statickom strome nebolo možné. Prvým krokom je nájdenie pozície, na ktorú tento kľúč patrí. To dosiahneme podobne ako v predošlej sekcii. Budeme ale hľadať predchádzajúci a nasledujúci kľúč. Tým nájdeme v usporiadanom poli dvojicu pozícii, medzi ktoré chceme vložiť novú hodnotu.

Vloženie do usporiadaného pola zmení súvislý interval veľkosti $K$. Následne bude potrebné aktualizovať kľúče v statickom strome, aby opäť obsahovali maximum zo svojich synov. Túto aktualizáciu dosiahneme takzvaným \emph{post-order} prechodom, kedy sa vrchol aktualizuje až po tom, čo boli aktualizovaný jeho synovia. Tým máme zaručené, že po aktualizácii vrchol obsahuje výslednú, korektnú hodnotu. Implementácia takéhoto prechodu je naznačená v algoritme \ref{alg:postorder}.

\begin{algorithm}
    \caption{Implementácia \emph{post-order} prechodu na aktualizáciu statického stromu}
    \label{alg:postorder}
    \begin{algorithmic}[1]
        \Function{update}{$v$, $I$} \Comment{$v$ je vrchol stromu, ktorý práve aktualizujeme}
        \Statex \Comment{$I$ je zmenený interval v usporiadanom poli}
        \Statex
            \If{$v$.podstrom $\cap~ I = \emptyset$} \Comment{pokiaľ sa zmena tohto vrcholu}
                \State \Return \Comment{určite nedotkla, tak ho môžeme preskočiť}
            \EndIf
            \Statex
            \If{$v$ je list}
                \AlgLet{$v$.kľuč}{hodnota z usporiadaného poľa}
            \Else
                \State \Call{update}{$v$.ľavý} \Comment{najskôr aktualizujeme ľavého}
                \State \Call{update}{$v$.pravý} \Comment{a pravého syna}
                \Statex
                \AlgLet{$v$.kľúč}{$\max(v$.ľavý.kľúč$, v$.pravý.kľúč$)$} \Comment{až potom tento vrchol}
            \EndIf           
        \EndFunction
    \end{algorithmic}
\end{algorithm}

\subsection{Analýza vkladania}

\begin{figure}
    \centering
    \resizebox{0.9\textwidth}{!}{
        \begin{tikzpicture}[xcenter around={(M0LT)}{(M3RB)}]

% Triangle
\newcommand{\width}{16}
\newcommand{\height}{8}
\newcommand{\hoff}{0.25}
\newcommand{\voff}{0.25}
% Inner
\newcommand{\hhoff}{\hoff*4}
\newcommand{\hheight}{\height*0.3}
% Center / LCA top
\newcommand{\choff}{-1}
\newcommand{\cvoff}{1}

% Memory layout box
\newcommand{\memoff}{0.25}
\newcommand{\memw}{2}
\newcommand{\memh}{1}

\newcommand{\extd}{0}
\newcommand{\braceoff}{0.05}

\newcommand{\trianglecenter}[4]{
  \coordinate (#1) at ($1/3*(#2)+1/3*(#3)+1/3*(#4)$);
}

% Big Left/Right/Top
\coordinate (BL) at (0, 0);
\coordinate (BR) at (\width, 0);
\coordinate (BT) at (\width/2, \height);
% Top Left/Right/Top
%\coordinate (TL) at (0.25*\width + \hoff, 0.5*\height);
%\coordinate (TR) at (0.75*\width - \hoff, 0.5*\height);
%\coordinate (TT) at ($ (BT)+(0, -\voff) $);
% Left Left/Right Right
\coordinate (LL) at ($ (\voff, \voff) + (\hhoff, 0) $);
\coordinate (RL) at ($ (\width - \voff, \voff) - (\hhoff, 0) $);
% Left+Right Center
\coordinate (LRC) at (0.5*\width, \voff);
\coordinate (LR) at ($ (LRC) - (0.5*\hoff, 0) $);
\coordinate (RR) at ($ (LRC) + (0.5*\hoff, 0) $);

\coordinate (TL) at ($ (LL)!0.5!(LR) + (0, \hheight) $);
\coordinate (TR) at ($ (RL)!0.5!(RR) + (0, \hheight) $);

% Centers
\trianglecenter{BC}{BL}{BR}{BT}
%\trianglecenter{TC}{TL}{TR}{TT}
\trianglecenter{LC}{LL}{LR}{TL}
\trianglecenter{RC}{RL}{RR}{TR}

% Outer triangle
\draw [ultra thick] (BL) -- (BR) -- (BT) -- (BL);
% Inner triangles
%\draw (TL) -- (TR) -- (TT) -- (TL);
\draw [thick] (LL) -- (TL) -- (LR) -- (LL);
\draw [thick] (RL) -- (TR) -- (RR) -- (RL);

%\node at (BC) {\Huge $\cdots$};
%\node at (TC) {\Huge $\tau_0$};
%\node at (LC) {\Huge $\tau_1$};
%\node at (RC) {\Huge $\tau_k$};

\coordinate (BM) at ($ (LC)!0.5!(RC) $);
\node at ($ (BM) + (0, \voff) $) {\Large $\cdots$};
%\node at ($ (BM)-(LC) + (RC)  $) {\Large $\cdots$};
%\node at ($ (BM)-(RC) + (LC)  $) {\Large $\cdots$};

% Inner
\newcommand{\innertriangles}[4]{
	\coordinate (#1TT) at ($ #2 - (0, \voff*0.5) $);
	\coordinate (#1LL) at ($ #3 + (\hoff, \voff*0.5) $);
	\coordinate (#1RR) at ($ #4 + (-\hoff, \voff*0.5) $);
	\coordinate (#1TL) at ($ (#1TT)!0.5!(#1LL) $);
	\coordinate (#1TR) at ($ (#1TT)!0.5!(#1RR) $);
	\coordinate (#1LRC) at ($ (#1LL)!0.5!(#1RR) $);
	\coordinate (#1LR) at ($ (#1LRC) - (0.25*\hoff, 0) $);
	\coordinate (#1RL) at ($ (#1LRC) + (0.25*\hoff, 0) $);
	
	%\draw (#1TT) -- (#1LL) -- (#1RR) -- (#1TT);
	
	\draw [thin] (#1TT) -- (#1TL) -- (#1TR) -- (#1TT);
	\draw [thin] (#1LL) -- (#1TL) -- (#1LR) -- (#1LL);
	\draw [thin] (#1RL) -- (#1TR) -- (#1RR) -- (#1RL);
	
	\trianglecenter{#1LC}{#1LL}{#1TL}{#1LR}
	\trianglecenter{#1RC}{#1RL}{#1TR}{#1RR}
	\trianglecenter{#1TC}{#1TT}{#1TR}{#1TL}
	\node at ($ (#1LC)!0.5!(#1RC) + (0, \voff*0.5) $) {$\cdots$};
	
	\node at (#1LC) {\small $\le B$};
	\node at (#1RC) {\small $\le B$};
	\node at (#1TC) {\small $\le B$};
}

\innertriangles{IL}{(TL)}{(LL)}{(LR)}
\innertriangles{IR}{(TR)}{(RR)}{(RL)}

\draw [thick, decorate, decoration={brace, amplitude=10pt, raise=1pt}] (LL) -- (TL) node [midway, anchor=south east, xshift=10pt, yshift=8pt] {$>B$};


% Center (LCA)
\coordinate (CT) at (\width*0.5+\choff, \height*0.5+\cvoff);
\draw [thick] (CT) -- (TL) -- (TR) -- (CT);

\trianglecenter{CC}{CT}{TL}{TR}
\node at (CC) {\Large $U$};

% LCA - root path
%\draw [->,ultra thick,]  ($(LB) + (-\arrangle:\arroff) $) to  [out=-\arrangle,in=180+\arrangle] node[below,midway]{\huge Zápis} ($(RB) + (180+\arrangle:\arroff) $);
\draw [thick, out=130, in=-70] (CT) to (BT);

% Vertical separators
\newdimen\XCoord
\newdimen\YCoord

\coordinate (S1L) at ($ (\height*0.5+\cvoff, \height*0.5+\cvoff) $);
\coordinate (S1R) at ($ (BR) - (S1L |- 0,0) + (S1L -| 0, 0) $);

\path (TL);
\pgfgetlastxy{\XCoord}{\YCoord}
\coordinate (S2L) at ($ (\YCoord, \YCoord) $);
\coordinate (S2R) at ($ (BR) - (S2L |- 0,0) + (S2L -| 0, 0) $);

\draw [dashed] ($ (S1L) - (\extd, 0) $)  -- ($ (S1R) + (\extd, 0) $);
\draw [dashed] ($ (S2L) - (\extd, 0) $)  -- (TL);
\draw [dashed] (TR)  -- ($ (S2R) + (\extd, 0) $);

\draw [thick, decorate, decoration={brace, amplitude=15pt, raise=1pt}] (BT) -- ($ (S1R) + (-\braceoff, \braceoff) $) node [midway, anchor=south west, xshift=10pt] {Horná časť};
\draw [thick, decorate, decoration={brace, amplitude=15pt, raise=1pt}] ($ (S1R) - (-\braceoff, \braceoff) $) -- ($ (S2R) + (-\braceoff, \braceoff) $) node [midway, anchor=south west, xshift=10pt] {Stredná časť};
\draw [thick, decorate, decoration={brace, amplitude=15pt, raise=1pt}] ($ (S2R) - (-\braceoff, \braceoff) $) -- (BR) node [midway, anchor=south west, xshift=10pt] {Spodné dve úrovne};


% Ordered file
\coordinate (M0LT) at ($ (BL) - (0, \memoff) $);
\coordinate (M0LB) at ($ (M0LT) - (0, \memh) $);
\coordinate (M0RB) at ($ (M0LT) + (\memw, -\memh) $);
\coordinate (M1RT) at ($ (M0RB) + (\memw, \memh) $);
\coordinate (M3RT) at ($ (BR) - (0, \memoff) $);
\coordinate (M3RB) at ($ (M3RT) - (0, \memh) $);
\coordinate (M3LB) at ($ (M3RT) - (\memw, \memh) $);
\coordinate (M4LB) at ($ (M3LB) - (\memw, 0) $);
\coordinate (M1RB) at ($ (M1RT) - (0, \memh) $);

\draw [ultra thick] (M0LT) rectangle (M3RB);
\draw (M0LT) rectangle (M1RB);
\draw [pattern=north west lines,] (M1RT) rectangle (M4LB);
\draw (M4LB) rectangle (M3RT);

\node [fill=white, align=center] at  ($ (M1RT)!0.5!(M4LB) $) {Upravený interval};
\draw [thick, decorate, decoration={brace, amplitude=15pt, raise=2pt, mirror}] (M1RB) -- (M4LB) node [midway, anchor=north, yshift=-15pt] {$K$};
\draw [thick, decorate, decoration={brace, amplitude=5pt, raise=2pt}] (M3RT) -- (M3RB) node [midway, anchor=west, xshift=5pt] {Usporiadané pole};

\draw [thick, decorate, decoration={brace, amplitude=15pt, raise=2pt, mirror}] (BT)--(BL) node [rotate=45, midway, anchor=south, yshift=15pt] {Statický strom};
\end{tikzpicture}
    }
    \caption{\todo[inline]{caption}}
    \label{fig:cobtree_analysis_parts}
\end{figure}

Túto analýzu rozdelíme na tri časti, v ktorých sa postupne pozrieme na rôzne úrovne v statickom strome, ktoré treba po vložení do usporiadaného pola aktualizovať.

\subsubsection{Spodné dve úrovne}
Uvažujme najskôr ako v časti \ref{sec:static-search} také podstromy \vEB delenia, ktoré sa ako prvé zmestia do bloku \cache a teda ich veľkosť je medzi $\sqrt{B}$ a $B$. Pozrime sa na spodné dve úrovne takýchto podstromov. Listy spodnej teda budú listy celého statického strom a korene spodnej spodnej úrovne budú synovia listov hornej úrovne. Zvyšok stromu pokračuje nad týmito úrovňami a vrátime sa k nemu neskôr.

Podstromy v tomto delení sa zmestia do najviac dvoch blokov a vieme ich teda načítať pomocou $\bigO{1}$ pamäťových presunov. Označme podstromy, ktoré vznikly rekurzívnym \vEB delením z podstromu $T$ veľkosti viac než $B$, rovnako, ako v časti \ref{sec:static-obliv}: $\tau_0, \tau_1, \dotsc, \tau_k$. Tu $\tau_0$ je podstrom v hornej úrovni a $\tau_1,\dotsc,\tau_k$ sú podstromy v spodnej úrovni, ktorých korene sú potomkovia listov $\tau_0$.

Pri \emph{post-order} prechode cez podstrom $T$ budeme prejdeme najskôr cez ľavých synov od koreňa $\tau_0$ cez $\tau_1$ až do listu. Následne bude \emph{post-order} prechod prebiehať v podstrome $\tau_1$, až kým nebudú všetky jeho vrcholy a napokon aj koreň aktualizované. Potom sa vrátime do $\tau_0$ a k $\tau_1$ už pristupovať nebudeme ale prejdeme nasledovný podstrom, $\tau_2$. Takto postupne aktualizujeme všetky podstromy, pričom postupnosť navštívených podstromov bude
\[
\tau_0, \tau_1, \tau_0, \tau_2, \dotsc, \tau_0, \tau_i, \tau_0, \dotsc, \tau_k, \tau_0
\]

Vidíme, že pokiaľ dokážeme udržať v \cache aspoň dva také podstromy, teda za predpokladu $M \ge 4B$, sme schopný takýto \emph{post-order} prechod zrealizovať pomocou $\bigO{1}$ pamäťových operácii pre každý podstrom veľkosti $\le B$. Takéto podstromy majú $\bigO{B}$ listov, pričom  nimi potrebujeme pokryť interval veľkosti $K$ (podstromy, ktorých žiaden list neleží v upravenom intervale nie je potrebné aktualizovať) a teda celkový počet podstromov na spodných dvoch úrovniach, ktoré potrebujeme načítať a upraviť je $\bigO{\frac{K}{B}}$ a stačí nám na to $\bigO{\frac{K}{B}}$ pamäťových presunov.

\subsubsection{Stredná úroveň - najbližší spoločný predok}
Označme ako $T_1,\dotsc,T_l$ tie stromy veľkosti $>B$ obsahujúce postromy veľkosti $\le B$ na spodných dvoch úrovniach, ktoré boli v predošlej časti aktualizované. Platí teda $l = \bigO{\frac{K}{B}}$. Označme $U$ taký podstrom statického stromu, ktorého koreň je najbližší spoločný predok koreňov $T_1,\dotsc,T_l$. Ak je tento koreň v hĺbke $h$ tak $U$ obsahuje všetky vrcholy hĺbky $\ge h$, ktoré treba aktualizovať - tie mimo $U$ nie sú predkovia upravených vrcholov a teda hodnoty kľúčov ich synov neboli v prvej časti zmenené.

Počet vrcholov $U$ je najviac dvojnásobok počtu listov a teda $|U| = \bigO{l} = \bigO{\frac{K}{B}}$. Keďže pri \emph{post-order} prechode navštívime každý vrchol $\bigO{1}$ krát (najprv pri prechode z koreňa ku listom, pričom následne rekurzívne prejdeme ľavého a pravého syna, a potom pri návrate z rekurzie) a teda celkovo budeme potrebovať $\bigO{\frac{K}{B}}$ pamäťových presunov na aktualizáciu $U$.

\subsubsection{Horná úroveň - cesta z najbližšieho spoločného predka do koreňa}
Posledná množina vrcholov, ktoré je potrebné aktualizovať je cesta z koreňa $U$ do koreňa statického stromu. Dĺžka tejto cesty je obmedzená výškou stromu $\bigO{\log{N}}$ a pri aktualizovaní prechádzame cez jej vrcholy postupne. Podobne ako pri vyhľadávaní (časť \ref{sec:dynamic-obliv-search}) bude potrebných $\bigO{\log_B{N}}$ pamäťových presunov.

\subsubsection{Výsledná zložitosť}
Sčítaním nasledovných zložitostí počtu pamäťových operácii
\[
\begin{aligned}
&\text{Nájdenie pozície v usporiadanom poli:} ~~ &&\bigO{\log_B{N}} \\
&\text{Vloženie do usporiadaného poľa:} &&\bigO{\frac{K}{B}} \\
&\text{Aktualizácia spodných dvoch úrovní:} &&\bigO{\frac{K}{B}} \\
&\text{Aktualizácia strednej úrovne:} &&\bigO{\frac{K}{B}} \\
&\text{Aktualizácia hornej úrovne:} &&\bigO{\log_B{N}} \\
\end{aligned}
\]
dostávame celkovú zložitosť $\bigO{\log_B{N} + \frac{K}{B}}$. Keďže amortizovaná veľkosť upraveného intervalu je $K = \bigO{\log^2{N}}$ dostávame výslednú amortizovanú zložitosť počtu pamäťových operácii pri vkladaní: $\bigO{\log_B{N} + \frac{\log^2{N}}{B}}$.

\subsection{Odstraňovanie}
Algoritmus odstraňovania je analogický, po nájdení prvku v usporiadanom poli ho odstránime, čím sa zmení súvislý amortizovanej veľkosti $\bigO{\log^2{N}}$. Následne aktualizujeme kľúče v statickom poli rovnako ako pri vkladaní. Výsledná zložitosť bude taktiež rovnaká.

\section{Vylepšený dynamický B-strom}
Oproti \aware B-stromom ma pri vkladaní \obliv dynamický strom popísaný v predchádzajúcej sekcii naviac $\bigO{\frac{\log^2{N}}{B}}$ pamäťových presunov. Odstrániť ho môžeme jednoduchou modifikáciou.

Vezmeme $N$ kľúčov a rozdelíme ich do $\Theta(\frac{N}{\log{N}})$ blokov veľkosti $\Theta(\log{N})$. Minimum z každej skupiny použijeme ako kľúče v predchádzajúcej dátovej štruktúre, ktorej veľkosť bude $\Theta(\frac{N}{\log{N}})$.

\subsection{Vyhľadávanie}
Najskôr vyhľadáme požadovaný blok v B-strome, čo zaberie $\bigO{\log_B{\frac{N}{\log{N}}}} = \bigO{\log_B{N}-\log_B{\log{N}}} = \bigO{\log_B{N}}$ pamäťových presunov. Následne prejdeme daný blok celý a nájdeme v ňom požadovaný kľúč. To vyžaduje $\bigO{\frac{\log{N}}{B}}$ pamäťových presunov - zanedbateľné voči hľadaniu v B-strome - a spolu teda vyhľadávanie vyžaduje rovnako veľa pamäťových presunov ako neupravený B-strom: $\bigO{\log_B{N}}$. \todo{popis najdenia bloku - min/max, left child, ...}

\subsection{Vkladanie a odstraňovanie}
Po nájdení požadovaného bloku vykonáme vloženie prípadne odstránenie z neho kompletným prepísaním, čo vyžaduje $\bigO{\frac{\log{N}}{B}}$ presunov. Zároveň budeme udržiavať veľkosti blokov medzi $\frac{1}{4}\log{N}$ a $\log{N}$. Príliš malé skupiny môžeme spojiť do väčších a veľké rozdeliť na niekoľko menších. Skupina prekročí svoje hranice najskôr po $\Omega(\log{N})$ operáciách. V takom prípade po rozdelení alebo spojení bude potrebné vykonať vloženie alebo odstránenie z B-stromu. Vydelením dostaneme amortizovanú zložitosť vkladania a odstraňovania v takto upravenej štruktúre: $\bigO{\log_B{N}}$.

Opäť sme teda dosiahli rovnaký počet operácií ako ekvivalentná \aware dátová štruktúra, avšak tentokrát v amortizovane.

\todo[inline]{jedine amortizovane je OF, to sa da worstcase, potom bude aj toto worstcase}

% OUTLINE
%- static search trees
%  - aware solution - btree
%  - naive solution - too slow
%  - obliv solution
%    - vEB layout
%      - indexing - pointers vs compute position
%    - analysis
%- ordered file
%  - desc
%  - analysis
%- dynamic search tree
%  - aware solution - btree again
%  - basic version
%    - analysis
%  - indirection
%    - analysis

%%%%%%%%%%%%%%%%%%%%%%%%%%%%%%%%%% OLD %%%%%%%%%%%%%%%%%%%%%%%%%%%%%%%%%%%%%%%%%%%%%%%%


%\section{Matice}
%
%\subsection{Násobenie matíc}
%
%Ďalším jednoduchým \obliv algoritmom je násobenie matíc. Majme dve matice $A, B$ typu $N \times N$ a chceme vypočítať ich súčin $S = A \cdot B$. Klasický algoritmus bude pri výpočte každého prvku $S$ postupne prechádzať maticu $A$ po riadkoch a maticu $B$ po stĺpoch. Za predpokladu, že sa do cache súčasne zmestia aspoň tri bloky - po jednom z matíc $A$ a $B$, a jeden blok $S$ obsahujúci prvok, ktorý práve počítame - budeme na každý prvok $S$ potrebovať najviac $\bigO{1+N/B}$ presunov. Celkovo teda vykonáme najviac $\bigO{N^2 + N^3/B}$ pamäťových presunov. 
%
%\
%
%\todo[inline]{Doplniť \obliv verziu + analýzu}
%
%%\subsection{Invertovanie matíc}
%
%\section{Stromy}
%\subsection{Statické stromy}
%
%\todo[inline]{...}
%
%%\subsubsection{Packed memory array}
%%\subsubsection{B-Stromy}
%
%%\subsubsection{Prioritné fronty} % funnel heap?
%
%\section{Triedenie}
%
%V \aware modely pamäte je spodným odhadom na počet pamäťových presunov pri triedení porovnávaním $\Theta(\frac{N}{B}log_{M/B}\frac{N}{B})$. \citep{AggVitt88, Demaine02}. Algoritmus, ktorý túto hranicu dosahuje je $M/B$-cestný mergesort (triedenie zlučovaním). Na rozdiel od klasického mergesortu si pri zlučovaní pamätá $B$ prvkov z každého z $M/B$ zoznamov a pri vyprázdnení načíta opäť cely blok s $B$ prvkami. Zlúčenie listov celkovej dĺžky $N$ teda vyžaduje $\bigO{N/B}$ pamäťových presunov.
%
%\
%
%Avšak \obliv algoritmy musia fungovať bez znalosti $M$ a $B$ a teda bez možnosti vypočítať $M/B$, najväčší počet zoznamov, ktoré môžeme súčasne zlučovať a pamätať si z každého $B$ prvkov v cache. Najlepšie, čo môžeme predpokladať je $M/B \ge 2$, teda vieme aspoň dva zoznamy zlučovať. Teda implementácia $2$-cestného mergesort algoritmu je \obliv, a funguje pre ľubovolné parametre. Avšak počet pamäťových presunov bude $\Theta(\frac{N}{B}log_{2}\frac{N}{B})$, to znamená, že zväčšenie $M$ tento algoritmus nezrýchli, keďže využívame vždy iba malú časť cache. 
%
%\
%
%Ideálny \obliv algoritmus by dosahoval rovnakú, optimálnu hranicu počtu presunov ako $M/B$-cestný mergesort, no bez znalosti týcho parametrov. Jedným z takýchto efektívnych \obliv algoritmov je takzvaný {\em funnel sort} - lievikové triedenie. Skôr ako ho môžeme popísať však potrebujeme definovať dátovú štruktúru {\em funnel} (lievik).
%
%\subsection{Funnel}
%
%K-lievik nazveme štruktúru, ktorá je na vstupe dostane $K$ usporiadaných zoznamov, s celkovou dĺžkou $K^3$ a skombinuje tieto prvky do jedného, usporiadaného výstupného zoznamu, pričom použije najviac $\bigO{\frac{K^3}{B} log_{M/B} \frac{K^3}{B} + K}$ pamäťových operácií.
%
%\
%
%Reprezentácia $K$-lievika bude úplný binárny strom s $K$ listami, uložený v pamäti vo van Emde Boasovom usporiadaní, ako pri statických stromoch (rekurzívne podstromy veľkosti $\sqrt{K}$). Hrany medzi vnútornými rekurzívnymi podstromami si uchovávajú {\em buffer} (pomocné pole) veľkosti $K^{3/2}$, pričom podstromov je $\sqrt{K}$ a teda spolu potrebujú $K^2$ pamäte. V podstromoch, ktoré tvoria $\sqrt{K}$-lieviky, sú všetky buffery rekurzívne menšie.
%
%\
%
%Spolu teda $K$-lievik potrebuje $S(K)$ pamäte. Každý sa skladá z $1+\sqrt{K}$ podstromov, ktoré reprezentujú $\sqrt{K}$-lieviky a teda $S(K) = (1+\sqrt{K})S(\sqrt{K}) + K^2$. Z toho jednoducho dostaneme, že veľkosť $K$-lievika v pamäti je $\bigO{K^2}$.
%
%\
%
%\todo[inline]{Obrázok + analýza počtu pamäťových presunov}
%
%\subsection{Funnelsort}
%
%Vezmime vstupné pole veľkosti $N$ a rozdelme ho na $K = N^{1/3}$ súvislých segmentov. Veľkosť každého bude $N^2/3$. Následne rekurzívne utriedime tieto segmenty. Pre ich spojenie použijeme $K$-lievik, ktorého výstupom bude usporiadané pole.
%
%\
%
%Počet pamäťových presunov bude
%\[
%T(N) = N^{1/3}T(N^{2/3}) + \bigO{\frac{N}{B} log_{M/B} \frac{N}{B} + N^{1/3}}
%\]
%keďže rozdelenie poľa je voči spájaniu zanedbateľné. Táto rekurencia platí pre $N > M$. V prípade, že sa cele pole zmestí do cache, teda $N \le M$, a za predpokladu $M \ge B^2$, dostávame $T(N) = T(B^2) = \bigO{B}$. Celkové riešenie tejto rekurencie, a teda výsledný počet pamäťových presunov potrebných na usporiadanie poľa veľkosti $N$ je $\bigO{\frac{N}{B} log_{M/B} \frac{N}{B}}$.




%\begin{figure}
%    \centering
%    
%    \subbottom[Abc] {
%        \resizebox{0.4\textwidth}{!}{
%            \begin{tikzpicture}

% Triangle
\newcommand{\width}{8}
\newcommand{\height}{4}
\newcommand{\hoff}{0.25}
\newcommand{\voff}{0.25}
% Memory layout box
\newcommand{\memoff}{0.5}
\newcommand{\memw}{1}
\newcommand{\memh}{1}


\newcommand{\trianglecenter}[4]{
  \coordinate (#1) at ($1/3*(#2)+1/3*(#3)+1/3*(#4)$);
}

% Big Left/Right/Top
\coordinate (BL) at (0, 0);
\coordinate (BR) at (\width, 0);
\coordinate (BT) at (\width/2, \height);
% Top Left/Right/Top
\coordinate (TL) at (0.25*\width + \hoff, 0.5*\height);
\coordinate (TR) at (0.75*\width - \hoff, 0.5*\height);
\coordinate (TT) at ($ (BT)+(0, -\voff) $);
% Left Left/Right Right
\coordinate (LL) at ($ (\voff, \voff) + (\hoff, 0) $);
\coordinate (RL) at ($ (\width - \voff, \voff) - (\hoff, 0) $);
% Left+Right Center
\coordinate (LRC) at (0.5*\width, \voff);
\coordinate (LR) at ($ (LRC) - (0.5*\hoff, 0) $);
\coordinate (RR) at ($ (LRC) + (0.5*\hoff, 0) $);
% Centers
\trianglecenter{BC}{BL}{BR}{BT}
\trianglecenter{TC}{TL}{TR}{TT}
\trianglecenter{LC}{LL}{LR}{TL}
\trianglecenter{RC}{RL}{RR}{TR}

% Outer triangle
\draw [thick] (BL) -- (BR) -- (BT) -- (BL);
% Inner triangles
\draw (TL) -- (TR) -- (TT) -- (TL);
\draw (LL) -- (TL) -- (LR) -- (LL);
\draw (RL) -- (TR) -- (RR) -- (RL);

\node at (BC) {\Huge $\cdots$};
\node at (TC) {\Huge $\tau_0$};
\node at (LC) {\Huge $\tau_1$};
\node at (RC) {\Huge $\tau_k$};

% Layout in memory
\coordinate (M0LT) at ($ (BL) - (0, \memoff) $);
\coordinate (M0RB) at ($ (M0LT) + (\memw, -\memh) $);
\coordinate (M1RT) at ($ (M0RB) + (\memw, \memh) $);
\coordinate (M3RT) at ($ (BR) - (0, \memoff) $);
\coordinate (M3LB) at ($ (M3RT) - (\memw, \memh) $);

\draw (M0LT) rectangle (M0RB);
\draw (M0RB) rectangle (M1RT);
\draw (M1RT) rectangle (M3LB);
\draw (M3LB) rectangle (M3RT);

\node at ($ (M0LT)!0.5!(M0RB) $) {\huge $\tau_0$};
\node at ($ (M0RB)!0.5!(M1RT) $) {\huge $\tau_1$};
\node at ($ (M1RT)!0.5!(M3LB) $) {\huge $\dots$};
\node at ($ (M3LB)!0.5!(M3RT) $) {\huge $\tau_k$};


\end{tikzpicture}
%        }
%    }
%    \subbottom[Def] {
%        \resizebox{0.4\textwidth}{!}{
%            \begin{tikzpicture}

% Triangle
\newcommand{\width}{8}
\newcommand{\height}{4}
\newcommand{\hoff}{0.25}
\newcommand{\voff}{0.25}
% Memory layout box
\newcommand{\memoff}{0.5}
\newcommand{\memw}{1}
\newcommand{\memh}{1}


\newcommand{\trianglecenter}[4]{
  \coordinate (#1) at ($1/3*(#2)+1/3*(#3)+1/3*(#4)$);
}

% Big Left/Right/Top
\coordinate (BL) at (0, 0);
\coordinate (BR) at (\width, 0);
\coordinate (BT) at (\width/2, \height);
% Top Left/Right/Top
\coordinate (TL) at (0.25*\width + \hoff, 0.5*\height);
\coordinate (TR) at (0.75*\width - \hoff, 0.5*\height);
\coordinate (TT) at ($ (BT)+(0, -\voff) $);
% Left Left/Right Right
\coordinate (LL) at ($ (\voff, \voff) + (\hoff, 0) $);
\coordinate (RL) at ($ (\width - \voff, \voff) - (\hoff, 0) $);
% Left+Right Center
\coordinate (LRC) at (0.5*\width, \voff);
\coordinate (LR) at ($ (LRC) - (0.5*\hoff, 0) $);
\coordinate (RR) at ($ (LRC) + (0.5*\hoff, 0) $);
% Centers
\trianglecenter{BC}{BL}{BR}{BT}
\trianglecenter{TC}{TL}{TR}{TT}
\trianglecenter{LC}{LL}{LR}{TL}
\trianglecenter{RC}{RL}{RR}{TR}

% Outer triangle
\draw [thick] (BL) -- (BR) -- (BT) -- (BL);
% Inner triangles
\draw (TL) -- (TR) -- (TT) -- (TL);
\draw (LL) -- (TL) -- (LR) -- (LL);
\draw (RL) -- (TR) -- (RR) -- (RL);

\node at (BC) {\Huge $\cdots$};
\node at (TC) {\Huge $\tau_0$};
\node at (LC) {\Huge $\tau_1$};
\node at (RC) {\Huge $\tau_k$};

% Layout in memory
\coordinate (M0LT) at ($ (BL) - (0, \memoff) $);
\coordinate (M0RB) at ($ (M0LT) + (\memw, -\memh) $);
\coordinate (M1RT) at ($ (M0RB) + (\memw, \memh) $);
\coordinate (M3RT) at ($ (BR) - (0, \memoff) $);
\coordinate (M3LB) at ($ (M3RT) - (\memw, \memh) $);

\draw (M0LT) rectangle (M0RB);
\draw (M0RB) rectangle (M1RT);
\draw (M1RT) rectangle (M3LB);
\draw (M3LB) rectangle (M3RT);

\node at ($ (M0LT)!0.5!(M0RB) $) {\huge $\tau_0$};
\node at ($ (M0RB)!0.5!(M1RT) $) {\huge $\tau_1$};
\node at ($ (M1RT)!0.5!(M3LB) $) {\huge $\dots$};
\node at ($ (M3LB)!0.5!(M3RT) $) {\huge $\tau_k$};


\end{tikzpicture}
%        }
%    }
%    \caption{Test}
%\end{figure}
