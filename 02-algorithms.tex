\chapter{Cache-oblivius algoritmy a dátové štruktúry}

\todo[inline]{Intro text - obsah kapitoly (alg/ds, analyza, ...)}

\section{Základné algoritmy}

Na demonštráciu \obliv algoritmov a ich analýzy v \extmem modeli použijeme jednoduchý algoritmus, ktorý počíta agregačnú funkciu nad hodnotami uloženými v poli.

\subsection{Popis algoritmu}

Majme pole $A$ veľkosti $|A|=N$ a označme jeho prvky $A = \{a_1,\dotsc,a_N\} \in X^N$. Chceme vypočítať hodnotu $f_g(A)$, kde $g: X \times Y \rightarrow Y$ je agregačná funkcia, $g_0 \in Y$ je počiatočná hodnota a $f_g: X^\infty \rightarrow Y$ je rozšírenie agregačnej funkcie definované následovne: \todo{Spravna notacia pre x infinity?}
\[
\begin{aligned}
f_g(\{a_1,\dotsc,a_k\}) &= g(a_k, f(\{a_1,\dotsc,a_{k-1}\})) \\
f_g(\emptyset) &= g_0
\end{aligned}
\]
% f(1, 2, 3) = g(3, f(1, 2)) = g(3, g(2, f(1))) = g(3, g(2, g(1, g0)))

Túto funkciu je možné implementovať jednoducho ako jeden cyklus. Schematickú verziu implementácie uvádzame v algoritme \ref{alg:aggregate}.

\begin{algorithm}
    \caption{Implementácia agregačnej funkcie $f_g$}
    \label{alg:aggregate}
    \begin{algorithmic}[1]
        \Function{aggregate}{$g, g_0, A$}
            \AlgLet{$y$}{$g_0$}
            \For{$i \gets 1,\dotsc,|A|$}
                \AlgLet{$y$}{$g(A[i], y)$}
            \EndFor
            \State \Return{$y$}
        \EndFunction
    \end{algorithmic}
\end{algorithm}

Tento algoritmus s použitím vhodnej funkcie $g$ a hodnoty $g_0$ je možné použiť na rôzne, často užitočné výpočty, ako napríklad maximum, minimum, suma a podobne:

\[
\begin{aligned}
g^{\textrm{max}}(x, y) &= \max(x, y) ~~ &g^{\textrm{max}}_0 &= -\infty \\
g^{\textrm{sum}}(x,y) &= x+y &g^{\textrm{sum}}_0 &= 0
\end{aligned}
\]


\subsection{Analýza zložitosti}
\subsubsection{Časová analýza}
Klasická časová analýza tohto algoritmu je triviálna ak uvažujeme \RAM. Keďže prístup ku každému prvku \inlcode{A[i]} zaberie konštantný čas a za predpokladu, že čas na výpočet funkcie $g$, $T_g$, je nezávislý na vstupe, bude výsledný čas na výpočet tejto funkcie

\[
T(N) = \bigO{1} + N[\bigO{1} + T_g + \bigO{1}] = T_g\cdot\bigO{N}
\]

\subsubsection{Pamäťová analýza}

V prípade \aware algoritmu by sme pole $A$ mali uložené v $\lceil \frac{N}{B} \rceil$ blokoch veľkosti $B$. Pri výpočte by sme postupne tieto bloky načítali do cache a pracovali s nimi. V rámci jedného bloku počas výpočtu nedochádza k pamäťovým presunom. Zároveň stačí každý prvok spracovať raz a teda celkový počet pamäťových operácií bude presne rovný počtu blokov, $\lceil \frac{N}{B} \rceil$. Tento algoritmus však požaduje znalosť parametra $B$ a explicitný presun blokov.

Jednoducho však vieme dosiahnuť (takmer) rovnakú zložitosť aj v prípade \obliv algoritmu \ref{alg:aggregate}, ktorý žiadne parametre pamäte zjavne nevyužíva a nepozná. Budeme predpokladať, že pole $A$ je uložené v súvislom úseku pamäte - to je možné dosiahnuť aj bez znalosti parametrov pamäte. Zvyšok algoritmu prebieha rovnako ako v predchádzajúcom prípade. Každý blok obsahujúci nejaký prvok poľa $A$ bude teda presunutý do cache práve raz, a žiadne iné presuny nenastanú. Ostáva zistiť, koľko takých blokov môže byť.

Keďže nepoznáme veľkosti blokov v pamäti, nevieme pri ukladaní prvkov poľa zaručiť zarovnanie so začiatkom bloku. V najhoršom prípade uložíme do prvého bloku iba jeden prvok. Potom bude nasledovať $\lfloor \frac{N}{B} \rfloor$ plných blokov a nakoniec ešte najviac jeden blok, ktorý opäť nie je plný. Spolu máme teda $\lfloor \frac{N}{B} \rfloor + 2$ blokov.

Pokiaľ $\lfloor \frac{N}{B} \rfloor < \lceil \frac{N}{B} \rceil$ máme spolu najviac $\lceil \frac{N}{B} \rceil + 1$ blokov. V opačnom prípade $B$ delí $N$, teda v prvom a poslednom bloku je spolu presne $B$ prvkov a medzi nimi sa nachádza najviac $\frac{N-B}{B} = \frac{N}{B} - 1$ plných. Teda blokov je vždy najviac $\lceil \frac{N}{B} \rceil +1$.

Zostrojili sme teda \obliv algoritmus s asymptoticky rovnakou zložitosťou $\bigO{\frac{N}{B}}$ ako optimálny \aware algoritmus, ktorého implementácia je však jednoduchšia, keďže nemusí explicitne spravovať presun blokov do cache.

\section{Vyhľadávacie stromy}

\todo[inline]{intro ...}
\todo[inline]{lowerbound}

\subsection{\Aware riešenie}
V prípade, že poznáme veľkosť blokov $B$ v cache, môžeme problém vyhľadávacích stromov riešiť B-stromom s vetvením $\Theta(B)$. Každý vrchol teda vieme načítať s použitím $\bigO{1}$ pamäťových presunov. Výška takého B-stromu, ktorý má $N$ listov, bude $\bigO{\log_B{N}}$. Celkovo teda vyhľadávanie v tomto strome vykoná $\bigO{\log_B{N}}$ pamäťových presunov. To zodpovedá dolnej hranici vo vete \todo{dokaz}.

\subsection{Naivné \obliv riešenie}
Predtým ako popíšeme efektívne \obliv riešenie, pozrime sa na klasický binárny vyhľadávací strom. Jednoduchý a častý spôsob ako usporiadať uzly binárneho stromu v pamäti je nasledovný. Koreň uložíme na pozíciu $1$. Ľavého a pravého potomka vrchola na pozícií $x$ uložíme na pozície $2x$ a $2x+1$. Otec vrcholu $x$ bude na pozícií $\lfloor\frac{x}{2}\rfloor$.

\todo[inline]{obrazok (rovnaky ako vEB layout ale ine cisla)}

Výhodou tohto usporiadania sú implicitné vzťahy medzi vrcholmi. Na udržiavanie stromu stačí jednorozmerné pole kľúčov. Na prechod medzi nimi môžeme použiť triviálne funkcie uvedene v \todo{algoritmus}.

% TODO caption/labels
% TODO side by side
%\begin{algorithm}
%    \caption{Left}
%    \begin{algorithmic}[1]
%        \Function{left}{$x$}
%            \State \Return{$2x$}
%        \EndFunction
%    \end{algorithmic}
%\end{algorithm}
%\begin{algorithm}
%    \caption{Right}
%    \begin{algorithmic}[1]
%        \Function{right}{$x$}
%            \State \Return{$2x+1$}
%        \EndFunction
%    \end{algorithmic}
%\end{algorithm}
%\begin{algorithm}
%    \caption{Parent}
%    \begin{algorithmic}[1]
%        \Function{parent}{$x$}
%            \State \Return{$\lfloor\frac{x}{2}\rfloor$}
%        \EndFunction
%    \end{algorithmic}
%\end{algorithm}

Nevýhodou je však vysoký počet pamäťových presunov pri vyhľadávaní. Výška tohto stromu je $\bigO{\log{N}}$. Pri načítaní vrcholu na pozícií $x$ sa v rovnakom bloku nachádzajú vrcholy na pozíciach
\[
x-k,\dotsc,x-1,x,x+1,\dotsc,x+l
\]
kde $k+l<B$. Pri ďalšom kroku vyhľadávania budeme potrebovať vrchol $2x$ alebo $2x+1$ a teda nás pozície menšie ako $x$ nezaujímajú. V najlepšom prípade teda bude $k=0$ a $l=B-1$. Aby sa v tomto intervale nachádzali požadované vrcholy musí platiť
\[
\begin{aligned}
2x+1 &\le x+l = x+B-1 \\
x &\le B-2
\end{aligned}
\]
To znamená, že pre pozície $x > B - 2$ už bude potrebný pamäťový presun pre každý vrchol. Vrchol s pozíciou $B-2$ bude mať hĺbku $O(\log{B})$ a teda počet vrcholov na ceste z koreňa do listu, ktorých pozície v pamäti sú väčšie ako $B-2$ bude $\Omega(\log{N}-\log{B})$. Pre každý z nich je potrebné vykonať pamäťový presun a teda vyhľadávanie v takto usporiadanom binárnom strome vykoná $\Omega(\log{\frac{N}{B}})$ pamäťových presunov, čo je horšie ako pri \aware B-strome.


%- static search trees
%  - aware solution - btree
%  - naive solution - too slow
%  - obliv solution
%    - vEB layout
%      - indexing - pointers vs compute position
%    - analysis
%- ordered file
%  - desc
%  - analysis
%- dynamic search tree
%  - aware solution - btree again
%  - basic version
%    - analysis
%  - indirection
%    - analysis

%%%%%%%%%%%%%%%%%%%%%%%%%%%%%%%%%% OLD %%%%%%%%%%%%%%%%%%%%%%%%%%%%%%%%%%%%%%%%%%%%%%%%


%\section{Matice}
%
%\subsection{Násobenie matíc}
%
%Ďalším jednoduchým \obliv algoritmom je násobenie matíc. Majme dve matice $A, B$ typu $N \times N$ a chceme vypočítať ich súčin $S = A \cdot B$. Klasický algoritmus bude pri výpočte každého prvku $S$ postupne prechádzať maticu $A$ po riadkoch a maticu $B$ po stĺpoch. Za predpokladu, že sa do cache súčasne zmestia aspoň tri bloky - po jednom z matíc $A$ a $B$, a jeden blok $S$ obsahujúci prvok, ktorý práve počítame - budeme na každý prvok $S$ potrebovať najviac $\bigO{1+N/B}$ presunov. Celkovo teda vykonáme najviac $\bigO{N^2 + N^3/B}$ pamäťových presunov. 
%
%\
%
%\todo[inline]{Doplniť \obliv verziu + analýzu}
%
%%\subsection{Invertovanie matíc}
%
%\section{Stromy}
%\subsection{Statické stromy}
%
%\todo[inline]{...}
%
%%\subsubsection{Packed memory array}
%%\subsubsection{B-Stromy}
%
%%\subsubsection{Prioritné fronty} % funnel heap?
%
%\section{Triedenie}
%
%V \aware modely pamäte je spodným odhadom na počet pamäťových presunov pri triedení porovnávaním $\Theta(\frac{N}{B}log_{M/B}\frac{N}{B})$. \citep{AggVitt88, Demaine02}. Algoritmus, ktorý túto hranicu dosahuje je $M/B$-cestný mergesort (triedenie zlučovaním). Na rozdiel od klasického mergesortu si pri zlučovaní pamätá $B$ prvkov z každého z $M/B$ zoznamov a pri vyprázdnení načíta opäť cely blok s $B$ prvkami. Zlúčenie listov celkovej dĺžky $N$ teda vyžaduje $\bigO{N/B}$ pamäťových presunov.
%
%\
%
%Avšak \obliv algoritmy musia fungovať bez znalosti $M$ a $B$ a teda bez možnosti vypočítať $M/B$, najväčší počet zoznamov, ktoré môžeme súčasne zlučovať a pamätať si z každého $B$ prvkov v cache. Najlepšie, čo môžeme predpokladať je $M/B \ge 2$, teda vieme aspoň dva zoznamy zlučovať. Teda implementácia $2$-cestného mergesort algoritmu je \obliv, a funguje pre ľubovolné parametre. Avšak počet pamäťových presunov bude $\Theta(\frac{N}{B}log_{2}\frac{N}{B})$, to znamená, že zväčšenie $M$ tento algoritmus nezrýchli, keďže využívame vždy iba malú časť cache. 
%
%\
%
%Ideálny \obliv algoritmus by dosahoval rovnakú, optimálnu hranicu počtu presunov ako $M/B$-cestný mergesort, no bez znalosti týcho parametrov. Jedným z takýchto efektívnych \obliv algoritmov je takzvaný {\em funnel sort} - lievikové triedenie. Skôr ako ho môžeme popísať však potrebujeme definovať dátovú štruktúru {\em funnel} (lievik).
%
%\subsection{Funnel}
%
%K-lievik nazveme štruktúru, ktorá je na vstupe dostane $K$ usporiadaných zoznamov, s celkovou dĺžkou $K^3$ a skombinuje tieto prvky do jedného, usporiadaného výstupného zoznamu, pričom použije najviac $\bigO{\frac{K^3}{B} log_{M/B} \frac{K^3}{B} + K}$ pamäťových operácií.
%
%\
%
%Reprezentácia $K$-lievika bude úplný binárny strom s $K$ listami, uložený v pamäti vo van Emde Boasovom usporiadaní, ako pri statických stromoch (rekurzívne podstromy veľkosti $\sqrt{K}$). Hrany medzi vnútornými rekurzívnymi podstromami si uchovávajú {\em buffer} (pomocné pole) veľkosti $K^{3/2}$, pričom podstromov je $\sqrt{K}$ a teda spolu potrebujú $K^2$ pamäte. V podstromoch, ktoré tvoria $\sqrt{K}$-lieviky, sú všetky buffery rekurzívne menšie.
%
%\
%
%Spolu teda $K$-lievik potrebuje $S(K)$ pamäte. Každý sa skladá z $1+\sqrt{K}$ podstromov, ktoré reprezentujú $\sqrt{K}$-lieviky a teda $S(K) = (1+\sqrt{K})S(\sqrt{K}) + K^2$. Z toho jednoducho dostaneme, že veľkosť $K$-lievika v pamäti je $\bigO{K^2}$.
%
%\
%
%\todo[inline]{Obrázok + analýza počtu pamäťových presunov}
%
%\subsection{Funnelsort}
%
%Vezmime vstupné pole veľkosti $N$ a rozdelme ho na $K = N^{1/3}$ súvislých segmentov. Veľkosť každého bude $N^2/3$. Následne rekurzívne utriedime tieto segmenty. Pre ich spojenie použijeme $K$-lievik, ktorého výstupom bude usporiadané pole.
%
%\
%
%Počet pamäťových presunov bude
%\[
%T(N) = N^{1/3}T(N^{2/3}) + \bigO{\frac{N}{B} log_{M/B} \frac{N}{B} + N^{1/3}}
%\]
%keďže rozdelenie poľa je voči spájaniu zanedbateľné. Táto rekurencia platí pre $N > M$. V prípade, že sa cele pole zmestí do cache, teda $N \le M$, a za predpokladu $M \ge B^2$, dostávame $T(N) = T(B^2) = \bigO{B}$. Celkové riešenie tejto rekurencie, a teda výsledný počet pamäťových presunov potrebných na usporiadanie poľa veľkosti $N$ je $\bigO{\frac{N}{B} log_{M/B} \frac{N}{B}}$.




%\begin{figure}
%    \centering
%    
%    \subbottom[Abc] {
%        \resizebox{0.4\textwidth}{!}{
%            \begin{tikzpicture}

% Triangle
\newcommand{\width}{8}
\newcommand{\height}{4}
\newcommand{\hoff}{0.25}
\newcommand{\voff}{0.25}
% Memory layout box
\newcommand{\memoff}{0.5}
\newcommand{\memw}{1}
\newcommand{\memh}{1}


\newcommand{\trianglecenter}[4]{
  \coordinate (#1) at ($1/3*(#2)+1/3*(#3)+1/3*(#4)$);
}

% Big Left/Right/Top
\coordinate (BL) at (0, 0);
\coordinate (BR) at (\width, 0);
\coordinate (BT) at (\width/2, \height);
% Top Left/Right/Top
\coordinate (TL) at (0.25*\width + \hoff, 0.5*\height);
\coordinate (TR) at (0.75*\width - \hoff, 0.5*\height);
\coordinate (TT) at ($ (BT)+(0, -\voff) $);
% Left Left/Right Right
\coordinate (LL) at ($ (\voff, \voff) + (\hoff, 0) $);
\coordinate (RL) at ($ (\width - \voff, \voff) - (\hoff, 0) $);
% Left+Right Center
\coordinate (LRC) at (0.5*\width, \voff);
\coordinate (LR) at ($ (LRC) - (0.5*\hoff, 0) $);
\coordinate (RR) at ($ (LRC) + (0.5*\hoff, 0) $);
% Centers
\trianglecenter{BC}{BL}{BR}{BT}
\trianglecenter{TC}{TL}{TR}{TT}
\trianglecenter{LC}{LL}{LR}{TL}
\trianglecenter{RC}{RL}{RR}{TR}

% Outer triangle
\draw [thick] (BL) -- (BR) -- (BT) -- (BL);
% Inner triangles
\draw (TL) -- (TR) -- (TT) -- (TL);
\draw (LL) -- (TL) -- (LR) -- (LL);
\draw (RL) -- (TR) -- (RR) -- (RL);

\node at (BC) {\Huge $\cdots$};
\node at (TC) {\Huge $\tau_0$};
\node at (LC) {\Huge $\tau_1$};
\node at (RC) {\Huge $\tau_k$};

% Layout in memory
\coordinate (M0LT) at ($ (BL) - (0, \memoff) $);
\coordinate (M0RB) at ($ (M0LT) + (\memw, -\memh) $);
\coordinate (M1RT) at ($ (M0RB) + (\memw, \memh) $);
\coordinate (M3RT) at ($ (BR) - (0, \memoff) $);
\coordinate (M3LB) at ($ (M3RT) - (\memw, \memh) $);

\draw (M0LT) rectangle (M0RB);
\draw (M0RB) rectangle (M1RT);
\draw (M1RT) rectangle (M3LB);
\draw (M3LB) rectangle (M3RT);

\node at ($ (M0LT)!0.5!(M0RB) $) {\huge $\tau_0$};
\node at ($ (M0RB)!0.5!(M1RT) $) {\huge $\tau_1$};
\node at ($ (M1RT)!0.5!(M3LB) $) {\huge $\dots$};
\node at ($ (M3LB)!0.5!(M3RT) $) {\huge $\tau_k$};


\end{tikzpicture}
%        }
%    }
%    \subbottom[Def] {
%        \resizebox{0.4\textwidth}{!}{
%            \begin{tikzpicture}

% Triangle
\newcommand{\width}{8}
\newcommand{\height}{4}
\newcommand{\hoff}{0.25}
\newcommand{\voff}{0.25}
% Memory layout box
\newcommand{\memoff}{0.5}
\newcommand{\memw}{1}
\newcommand{\memh}{1}


\newcommand{\trianglecenter}[4]{
  \coordinate (#1) at ($1/3*(#2)+1/3*(#3)+1/3*(#4)$);
}

% Big Left/Right/Top
\coordinate (BL) at (0, 0);
\coordinate (BR) at (\width, 0);
\coordinate (BT) at (\width/2, \height);
% Top Left/Right/Top
\coordinate (TL) at (0.25*\width + \hoff, 0.5*\height);
\coordinate (TR) at (0.75*\width - \hoff, 0.5*\height);
\coordinate (TT) at ($ (BT)+(0, -\voff) $);
% Left Left/Right Right
\coordinate (LL) at ($ (\voff, \voff) + (\hoff, 0) $);
\coordinate (RL) at ($ (\width - \voff, \voff) - (\hoff, 0) $);
% Left+Right Center
\coordinate (LRC) at (0.5*\width, \voff);
\coordinate (LR) at ($ (LRC) - (0.5*\hoff, 0) $);
\coordinate (RR) at ($ (LRC) + (0.5*\hoff, 0) $);
% Centers
\trianglecenter{BC}{BL}{BR}{BT}
\trianglecenter{TC}{TL}{TR}{TT}
\trianglecenter{LC}{LL}{LR}{TL}
\trianglecenter{RC}{RL}{RR}{TR}

% Outer triangle
\draw [thick] (BL) -- (BR) -- (BT) -- (BL);
% Inner triangles
\draw (TL) -- (TR) -- (TT) -- (TL);
\draw (LL) -- (TL) -- (LR) -- (LL);
\draw (RL) -- (TR) -- (RR) -- (RL);

\node at (BC) {\Huge $\cdots$};
\node at (TC) {\Huge $\tau_0$};
\node at (LC) {\Huge $\tau_1$};
\node at (RC) {\Huge $\tau_k$};

% Layout in memory
\coordinate (M0LT) at ($ (BL) - (0, \memoff) $);
\coordinate (M0RB) at ($ (M0LT) + (\memw, -\memh) $);
\coordinate (M1RT) at ($ (M0RB) + (\memw, \memh) $);
\coordinate (M3RT) at ($ (BR) - (0, \memoff) $);
\coordinate (M3LB) at ($ (M3RT) - (\memw, \memh) $);

\draw (M0LT) rectangle (M0RB);
\draw (M0RB) rectangle (M1RT);
\draw (M1RT) rectangle (M3LB);
\draw (M3LB) rectangle (M3RT);

\node at ($ (M0LT)!0.5!(M0RB) $) {\huge $\tau_0$};
\node at ($ (M0RB)!0.5!(M1RT) $) {\huge $\tau_1$};
\node at ($ (M1RT)!0.5!(M3LB) $) {\huge $\dots$};
\node at ($ (M3LB)!0.5!(M3RT) $) {\huge $\tau_k$};


\end{tikzpicture}
%        }
%    }
%    \caption{Test}
%\end{figure}
