\chapter{Cache-oblivius algoritmy a dátové štruktúry}

%%%%%%%%%%%%%%%%%%%%%%%%%%%%%%%%%%%%%%%% OLD %%%%%%%%%%%%%%%%%%%%%%%%%%%%%%%%%%%%%%%
\section{Základné algoritmy}

Najjednoduchším \obliv algoritmom je prechod poľa a výpočet agregačnej funkcie. Majme pole $A$ veľkosti $|A| = N$ a označme jeho prvky $A_1,\cdots,A_n$. Chceme vypočítať hodnotu $f(g, A)$, kde $g$ je agregačná funkcia, $g_0$ je počiatočná hodnota a $f$ je definovaná následovne:
\[
\begin{aligned}
f(A_1,\cdots,A_n) &= g(A_1, f(A_2,\cdots,A_n)) \\
f(\emptyset) &= g_0
\end{aligned}
\]

Tento algoritmus s použitím vhodnej funkcie $g$ a hodnoty $g_0$ je možné použiť na rôzne, často užitočné výpočty, ako napríklad maximum ($g(x, y) = max(x, y)$, $g_0 = -\infty$), minimum, suma ($g(x,y) = x+y$, $g_0 = 0$) a podobne.

\

V prípade \aware algoritmu by sme pole $A$ mali uložené v $\lceil N/B \rceil$ blokoch veľkosti $B$. Pri výpočte $f$ by sme postupne tieto bloky načítali do cache a pracovali s nimi. V rámci jedného bloku počas výpočtu nedochádza k pamäťovým presunom. Zároveň stačí každý prvok spracovať raz a teda celkový počet pamäťových operácií bude presne rovný počtu blokov, $\lceil N/B \rceil$. Tento algoritmus však požaduje znalosť parametra $B$.

\

Jednoducho však vieme dosiahnuť (takmer) rovnakú zložitosť aj v prípade \obliv. Uložíme si pole $A$ do súvislého úseku pamäte - k tomu nepotrebujeme poznať hodnotu $B$ ani žiadne iné parametre. Zvyšok algoritmu prebieha rovnako ako v predchádzajúcom prípade. Každý blok obsahujúci nejaký prvok poľa $A$ bude teda presunutý do cache práve raz, a žiadne iné presuny nenastanú. Ostáva zistiť, koľko takých blokov môže byť.

\

Keďže nepoznáme veľkosti blokov v pamäti, nevieme pri ukladaní prvkov poľa zaručiť zarovnanie so začiatkom bloku. V najhoršom prípade uložíme do prvého bloku iba jeden prvok. Potom bude nasledovať $\lfloor N/B \rfloor$ plných blokov a nakoniec ešte najviac jeden blok, ktorý opäť nie je plný. Spolu máme teda $\lfloor N/B \rfloor + 2$. Pokiaľ $\lfloor N/B \rfloor < \lceil N/B \rceil$ a teda najviac $\lceil N/B \rceil + 1$ blokov. V opačnom prípade $B$ delí $N$, teda v prvom a poslednom bloku je spolu presne $B$ prvkov a medzi nimi sa nachádza najviac $\frac{N-B}{B} = N/B - 1$ plných. Teda blokov je vždy najviac $\lceil N/B \rceil +1$, čo je až aditívnu konštantu rovnaký výsledok ako uvedený \aware algoritmus.

\section{Matice}

\subsection{Násobenie matíc}

Ďalším jednoduchým \obliv algoritmom je násobenie matíc. Majme dve matice $A, B$ typu $N \times N$ a chceme vypočítať ich súčin $S = A \cdot B$. Klasický algoritmus bude pri výpočte každého prvku $S$ postupne prechádzať maticu $A$ po riadkoch a maticu $B$ po stĺpoch. Za predpokladu, že sa do cache súčasne zmestia aspoň tri bloky - po jednom z matíc $A$ a $B$, a jeden blok $S$ obsahujúci prvok, ktorý práve počítame - budeme na každý prvok $S$ potrebovať najviac $\bigO{1+N/B}$ presunov. Celkovo teda vykonáme najviac $\bigO{N^2 + N^3/B}$ pamäťových presunov. 

\

\todo[inline]{Doplniť \obliv verziu + analýzu}

%\subsection{Invertovanie matíc}

\section{Stromy}
\subsection{Statické stromy}

\todo[inline]{...}

%\subsubsection{Packed memory array}
%\subsubsection{B-Stromy}

%\subsubsection{Prioritné fronty} % funnel heap?

\section{Triedenie}

V \aware modely pamäte je spodným odhadom na počet pamäťových presunov pri triedení porovnávaním $\Theta(\frac{N}{B}log_{M/B}\frac{N}{B})$. \citep{AggVitt88, Demaine02}. Algoritmus, ktorý túto hranicu dosahuje je $M/B$-cestný mergesort (triedenie zlučovaním). Na rozdiel od klasického mergesortu si pri zlučovaní pamätá $B$ prvkov z každého z $M/B$ zoznamov a pri vyprázdnení načíta opäť cely blok s $B$ prvkami. Zlúčenie listov celkovej dĺžky $N$ teda vyžaduje $\bigO{N/B}$ pamäťových presunov.

\

Avšak \obliv algoritmy musia fungovať bez znalosti $M$ a $B$ a teda bez možnosti vypočítať $M/B$, najväčší počet zoznamov, ktoré môžeme súčasne zlučovať a pamätať si z každého $B$ prvkov v cache. Najlepšie, čo môžeme predpokladať je $M/B \ge 2$, teda vieme aspoň dva zoznamy zlučovať. Teda implementácia $2$-cestného mergesort algoritmu je \obliv, a funguje pre ľubovolné parametre. Avšak počet pamäťových presunov bude $\Theta(\frac{N}{B}log_{2}\frac{N}{B})$, to znamená, že zväčšenie $M$ tento algoritmus nezrýchli, keďže využívame vždy iba malú časť cache. 

\

Ideálny \obliv algoritmus by dosahoval rovnakú, optimálnu hranicu počtu presunov ako $M/B$-cestný mergesort, no bez znalosti týcho parametrov. Jedným z takýchto efektívnych \obliv algoritmov je takzvaný {\em funnel sort} - lievikové triedenie. Skôr ako ho môžeme popísať však potrebujeme definovať dátovú štruktúru {\em funnel} (lievik).

\subsection{Funnel}

K-lievik nazveme štruktúru, ktorá je na vstupe dostane $K$ usporiadaných zoznamov, s celkovou dĺžkou $K^3$ a skombinuje tieto prvky do jedného, usporiadaného výstupného zoznamu, pričom použije najviac $\bigO{\frac{K^3}{B} log_{M/B} \frac{K^3}{B} + K}$ pamäťových operácií.

\

Reprezentácia $K$-lievika bude úplný binárny strom s $K$ listami, uložený v pamäti vo van Emde Boasovom usporiadaní, ako pri statických stromoch (rekurzívne podstromy veľkosti $\sqrt{K}$). Hrany medzi vnútornými rekurzívnymi podstromami si uchovávajú {\em buffer} (pomocné pole) veľkosti $K^{3/2}$, pričom podstromov je $\sqrt{K}$ a teda spolu potrebujú $K^2$ pamäte. V podstromoch, ktoré tvoria $\sqrt{K}$-lieviky, sú všetky buffery rekurzívne menšie.

\

Spolu teda $K$-lievik potrebuje $S(K)$ pamäte. Každý sa skladá z $1+\sqrt{K}$ podstromov, ktoré reprezentujú $\sqrt{K}$-lieviky a teda $S(K) = (1+\sqrt{K})S(\sqrt{K}) + K^2$. Z toho jednoducho dostaneme, že veľkosť $K$-lievika v pamäti je $\bigO{K^2}$.

\

\todo[inline]{Obrázok + analýza počtu pamäťových presunov}

\subsection{Funnelsort}

Vezmime vstupné pole veľkosti $N$ a rozdelme ho na $K = N^{1/3}$ súvislých segmentov. Veľkosť každého bude $N^2/3$. Následne rekurzívne utriedime tieto segmenty. Pre ich spojenie použijeme $K$-lievik, ktorého výstupom bude usporiadané pole.

\

Počet pamäťových presunov bude
\[
T(N) = N^{1/3}T(N^{2/3}) + \bigO{\frac{N}{B} log_{M/B} \frac{N}{B} + N^{1/3}}
\]
keďže rozdelenie poľa je voči spájaniu zanedbateľné. Táto rekurencia platí pre $N > M$. V prípade, že sa cele pole zmestí do cache, teda $N \le M$, a za predpokladu $M \ge B^2$, dostávame $T(N) = T(B^2) = \bigO{B}$. Celkové riešenie tejto rekurencie, a teda výsledný počet pamäťových presunov potrebných na usporiadanie poľa veľkosti $N$ je $\bigO{\frac{N}{B} log_{M/B} \frac{N}{B}}$.