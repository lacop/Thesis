\chapter{Pamäťový model}
\epigraph{Memory is the thing you forget~with.}{Alexander Chase}

Pri časovej analýze algoritmov sa zvyčajne používa takzvaný \RAM (skratka z anglického \emph{Random-Access Machine}, stroj s náhodným prístupom k pamäti) \citep{aho1974design}, v ktorom sa predpokladá možnosť pristupovať k ľubovoľnému úseku pamäte v konštantnom čase. To znamená, že vo výslednej asymptotickej analýze počítame len počet vykonaných operácií.

V skutočnosti však moderné počítače využívajú niekoľko úrovňovú pamäťovú hierarchiu \citep{drepper2007every}. Tá sa typicky skladá z registrov a troch úrovní \cache (vyrovnávacej pamäte) priamo na procesore, následne z hlavnej operačnej pamäte a disku. V tomto poradí sú tieto úrovne zoradené od najrýchlejšej a najmenšej (odozva rádovo \SI{1}{\nano\second}, kapacita \SI{64}{\kibi\byte}) až po najpomalšiu ale najväčšiu (odozva od \SI{100}{\micro\second} po \SI{10}{\milli\second} podľa typu\footnote{Klasické pevné disky (HDD) alebo disky bez pohyblivých častí (SSD)}, kapacita rádovo \SI{1}{\tebi\byte}). Približné hodnoty pre všetky úrovne sú v tabuľke \ref{tbl:memory-levels}. Ako vidieť, rozdiel v prístupovej dobe k rôznym dátam môže byť až $10\,000\,000$-násobný. Naskytuje sa teda otázka, či je užitočné a presné hovoriť o konštantnom čase.

\begin{table}
    \centering
    \caption[Približné parametre rôznych úrovní pamäte]{Približné parametre rôznych úrovní pamäte. Hodnoty troch úrovní \cache na procesore (L1, L2 a L3) uvádzame pre mikroarchitektúru Intel Haswell. \citep{inteloptimize,inteldeveloper}}
    \label{tbl:memory-levels}
    \begin{threeparttable}
        \begin{tabular}{r r r l c r}
            \toprule
            \multicolumn{1}{c}{\textbf{Úroveň}} & \multicolumn{1}{c}{\textbf{Veľkosť}} & \multicolumn{2}{c}{\textbf{Odozva}} & \multicolumn{1}{c}{\textbf{Asociativita}} & \multicolumn{1}{c}{\textbf{Veľkosť bloku}}\\ \toprule
%            Registre & & 1clk\tnote{2} &$\approx$ \SI{0.3}{\nano\second} \\ \midrule
            L1 & \SI{64}{\kibi\byte}\tnote{1} & 4clk\tnote{2} &$\approx$ \SI{1}{\nano\second} & 8 & \SI{64}{\byte} \\
            L2 & \SI{256}{\kibi\byte}\tnote{1} & 11clk\tnote{2} &$\approx$ \SI{4}{\nano\second} &  8 & \SI{64}{\byte} \\
            L3 & \SI[parse-numbers=false]{2\mbox{--}20}{\mebi\byte} & 36clk\tnote{2} &$\approx$ \SI{12}{\nano\second} & & \SI{64}{\byte} \\ \midrule
            RAM & $\approx$ \SI{8}{\gibi\byte} & &$\approx$ \SI{100}{\nano\second} & & \SI{16}{\byte}\\
            Disk & $\approx$ \SI{1}{\tebi\byte} & &$\approx$ \SI[parse-numbers=false]{0.1\mbox{--}10}{\milli\second} & & \SI{4}{\kibi\byte}--\SI{2}{\mebi\byte} \\
            \bottomrule
        \end{tabular}
        \begin{tablenotes}
            \item[1] Hodnota pre jedno jadro procesora.
            \item[2] Počet cyklov procesora, uvedené časové aproximácie pri \SI{3}{\giga\hertz}.
        \end{tablenotes}
    \end{threeparttable}
\end{table}
%\todo[inline]{ram blok - dobre?}
%\todo[inline]{registre? tazko zratat velkost}

% Praktickým dôvodom pre túto hierarchiu je cena a jednoduchosť výroby.
% TODO: 1ns * c = 30cm, limit

Dôsledkom tejto hierarchie je závislosť výslednej rýchlosti algoritmu od jeho prístupov k pamäti. Operácie, ktoré využívajú dáta uložené na disku potrvajú dlhšie ako tie, ktoré využívajú iba dáta v registroch. Pri prístupe k dátam na disku sa v skutočnosti tieto dáta postupne presunú z disku do hlavnej pamäte, do \cache na procesore a napokon do registrov. Toto zabezpečí, že ich opakované použitie, pokiaľ nebudú dovtedy z \cache odstránené, bude rýchlejšie. Pre všetky susedné dvojice pamäťových úrovní teda slúži tá menšia a rýchlejšia ako vyrovnávacia pamäť pre tú väčšiu a pomalšiu.

\section{External-memory model} \label{sec:extmem}
Jedným zo spôsobov, ako zohľadniť tieto skutočnosti pri analýze algoritmov, je takzvaný \emph{external-memory model} (model externej pamäte), nazývaný tiež \emph{I/O model} alebo \emph{cache-aware model} \citep{aggarwal1988input}. Ten popisuje pamäť skladajúcu sa z dvoch častí (obrázok \ref{fig:exmem_model}), ktoré voláme \cache a \disk.

Všetky výpočty prebiehajú nad dátami v \cache, ktorá má obmedzenú veľkosť. Ostatné dáta sú uložené na disku neobmedzenej veľkosti, no nemôžeme s nimi priamo manipulovať a je ich potrebné najskôr preniesť do \cache. V samotnej analýze algoritmov potom počítame počet týchto prenosov z disku do \cache a naopak.

Prístup k dátam, ktoré sa práve nachádzajú v \cache voláme \hit (zásah \cache). Naopak prístup, ktorý vyžaduje dáta najskôr presunúť z \disk-u do \cache voláme \miss (minutie \cache). Tieto presuny sú realizované v \emph{blokoch} pamäte veľkosti $B$. Disk aj \cache sa skladajú z takýchto blokov a za jednu operáciu považujeme presun jedného bloku medzi nimi. \Cache má obmedzenú veľkosť $M$ a skladá sa teda z $\frac{M}{B}$ blokov.  

\begin{figure}
    \centering
    \resizebox{0.8\textwidth}{!}{%
        
\begin{tikzpicture}

\newcommand{\width}{4}
\newcommand{\height}{1}
\newcommand{\cacheheight}{8}
\newcommand{\cachebox}{\height}
\newcommand{\cachewidth}{4} % TODO divide?
\newcommand{\off}{1}
\newcommand{\diskoff}{3*\width}

\newcommand{\jaggedsep}{0.25}

\newcommand{\arrangle}{20}
\newcommand{\arroff}{1}

\foreach \i in {1,...,\cacheheight}
{
\draw [thick] (0, \i*\height-\height) rectangle (\width, \i*\height);
}

\foreach \i in {1,...,\cachewidth}
{
\draw [thick] (\i*\cachebox-\cachebox, 0) rectangle (\i*\cachebox, \height);
}

\draw [thick, |-|] (0,-\off) -- (\width, -\off) node [midway, fill=white,minimum width=2cm] {\Huge $B$};
\draw [thick, |-|] (-\off,0) -- ++ ($(0, \cacheheight*\height) $) node [midway, fill=white, minimum height=2cm] {\Huge $\frac{M}{B}$};

% Dont draw bottom edge
\foreach \i in {2,...,\cacheheight}
{
\draw [thick](\diskoff, \i*\height-\height) rectangle (\diskoff+\width, \i*\height);
}
\draw [thick] (\diskoff, 0) -- (\diskoff, \height);
\draw [thick] (\diskoff+\width, 0) -- (\diskoff+\width, \height);
%\raggedline{(\diskoff, 0) -- (\diskoff+\width, 0)}
\draw [thick] ($(\diskoff-0.5, -0.5)$) -- ++(1, 1) -- ++(1, -1) -- ++(1, 1) -- ++(1, -1) -- ($(\diskoff+\width+0.5, 0.5)$);
\draw [thick] ($(\diskoff-0.5, -0.5-\jaggedsep)$) -- ++(1, 1) -- ++(1, -1) -- ++(1, 1) -- ++(1, -1) -- ($(\diskoff+\width+0.5, 0.5-\jaggedsep)$);

\draw [thick, |-] (\diskoff+\width+\off,\cacheheight*\height) -- ++ ($(0, -\cacheheight*\height+\height) $) node [midway, fill=white, minimum height=1cm] {\Huge $\infty$} node [pos=1, fill=white] {\Huge $\vdots$};

\coordinate (BBL) at (0.5*\diskoff, 0.5*\height*\cacheheight);
\coordinate (BTR) at ($ (BBL) +(\width, \height) $);
\coordinate (BC) at ($ 0.5*(BBL) + 0.5*(BTR) $);
\draw [thick] (BBL) rectangle (BTR);

\node at (0.5*\width, \height*\cacheheight+\off) {\Huge Cache};
\node at (0.5*\width+\diskoff, \height*\cacheheight+\off) {\Huge Disk};
%\node at ($ (BC) + (0, 0.5*\height+\off) $) {\Huge Presúvaný blok};
\node at (BC) {\Large Presúvaný blok};

\coordinate (LB) at (\width, 0.5*\height*\cacheheight);
\coordinate (RB) at ($ (LB) + (\diskoff-\width, 0) $);
\coordinate (LT) at ($ (LB) + (0, \height) $);
\coordinate (RT) at ($ (RB) + (0, \height) $);

%\draw [->,thick,decoration={markings,mark=at position 1 with {\arrow[scale=2]{>}}},postaction={decorate}]  ($(LB) + (-\arrangle:\arroff) $) to  [out=-\arrangle,in=180+\arrangle] ($(RB) + (180+\arrangle:\arroff) $);
%\draw [->,thick,decoration={markings,mark=at position 1 with {\arrow[scale=2]{>}}},postaction={decorate}]  ($(RT) + (180-\arrangle:\arroff) $) to  [out=180-\arrangle,in=\arrangle] ($(LT) + (\arrangle:\arroff) $);

\draw [->,ultra thick,]  ($(LB) + (-\arrangle:\arroff) $) to  [out=-\arrangle,in=180+\arrangle] node[below,midway]{\huge Zápis} ($(RB) + (180+\arrangle:\arroff) $);
\draw [->,ultra thick]  ($(RT) + (180-\arrangle:\arroff) $) to  [out=180-\arrangle,in=\arrangle] node[above,midway]{\huge Čítanie} ($(LT) + (\arrangle:\arroff) $);


\end{tikzpicture}    
    }
    \caption{External-memory model}
    \label{fig:exmem_model}
\end{figure}

\subsection{Cache-aware algoritmy}
Pokiaľ poznáme parametre $B$ a $M$, môžeme skonštruovať algoritmus, ktorý bude túto dvojicu pamätí využívať efektívne. Takýto algoritmus voláme \aware (uvedomujúci si \cache). Súčasťou tohto algoritmu by bolo spravovanie presunov pamäte -- je potrebné riešiť čítanie blokov z disku a ich umiestňovanie do \cache, nahrádzanie blokov v \cache pri zaplnení a spätný zápis blokov na disk. Toto nemusí byť explicitnou súčasťou algoritmu a môže byť riešené na inej úrovni.

% TODO asociativita

Tento model popisuje len dve úrovne pamäte a teda funguje efektívne len pre danú susednú dvojicu, pre ktorú ho na základe znalosti parametrov optimalizujeme. V~moderných systémoch ale máme takýchto dvojíc niekoľko. Keby sme poznali parametre pre všetky tieto dvojice, môžeme tieto algoritmy optimalizovať pre všetky susedné dvojice a ich parametre. Stále však zostáva problémom viazanosť algoritmu na tieto parametre a pri ich zmene prestáva byť optimálny.


\section{Cache-oblivious model}
Druhým spôsobom, ktorý zohľadňuje nekonštantný prístupový čas k údajom v pamäti, je takzvaný \emph{cache-oblivious model} (na \cache nedbajúci), v ktorom uvažujeme rovnakú dvoj-úrovňovú pamäť zloženú z disku a \cache \citep{frigo1999cache,prokop1999cache}. Na rozdiel od \aware modelu však algoritmus nepozná parametre $B$ a $M$. Pokiaľ sa nám napriek tomu podarí navrhnúť algoritmus, ktorý vykonáva (asymptoticky) rovnaký počet pamäťových presunov ako \aware algoritmus, bude bežať efektívne pre ľubovolné takéto parametre.

%Takéto algoritmy majú na rozdiel od \aware algoritmov v \extmem modeli mnohé výhody. Samotná implementácia algoritmu nemôže explicitne riešiť presun blokov pokiaľ nepozná veľkosť bloku ani koľko blokov môže do \cache uložiť. Táto úloha zostane ponechaná na nižšiu vrstvu (operačný systém respektíve hardvér v prípade \cache na procesore) - algoritmus bude pristupovať k pamäti priamo bez ohľadu na to, či sa nachádza v \cache alebo nie a v prípade potreby prebehnú nutné prenosy na nižšej úrovni (z pohľadu algoritmu) automaticky.

Výhodou oproti \aware algoritmom je automatická optimalizácia pre dané parametre. V prípade \aware algoritmov môže byť problémom získať presné hodnoty týchto parametrov a potrebné pri ich zmene upraviť algoritmus. Vývoj algoritmu, ktorý bude fungovať na rozličných architektúrach, môže byť značne problematický.

Ďalšou výhodou je, že takýto \obliv algoritmus bude (rovnako) efektívny medzi každou dvojicou susedných úrovní. Vzhľadom na to, že hodnoty parametrov nepozná, bude pre ľubovolnú takú dvojicu pracovať rovnako efektívne ako pre každú inú.

% TODO assumptions (replacement + associativity)
% TODO \cache vs cache (+ cache-obliv, ... ) - when to use \emph?

\subsection{Správa pamäte} \label{sec:memmng}
V momente, keď sa \obliv algoritmus pokúsi o vykonanie operácie, ktorá potrebuje dáta mimo \cache, je potrebné ich najskôr z disku skopírovať. V prípade, že je v cache voľný blok, je možné presunúť dáta bez nutnosti nahradenia. V opačnom prípade je však potrebné uvoľniť miesto tým, že sa vyberie blok z \cache (ak bol tento blok upravený, najskôr sa jeho obsah zapíše späť na disk), ktorý bude následne prepísaný požadovaným blokom. Tento proces sa nazýva výmena stránok ({\em page replacement}), a algoritmus rozhodujúci, ktorý blok z cache odstrániť, voláme stratégia výmeny stránok ({\em page-replacement strategy}). Dve základné stratégie výmeny stránok sú \emph{LRU} a \emph{FIFO}.

Stratégia \emph{LRU} (least recently used -- najdlhšie nepoužité) vyberá vždy blok, ktorý bol najdlhšie nepoužitý. Implementácia vyžaduje udržiavať si ku každému bloku počítadlo, ktoré sa pri prístupe nastaví na nulu a pri prístupe k iným blokom zvýši o jedna. Pri potrebe uvoľniť miesto v cache vyberieme blok s najväčšou hodnotou počítadla - ten, ku ktorému najdlhšie nebol prístup.

Stratégia \emph{FIFO} (first in, first out -- prvé dnu, prvé von) je ešte jednoduchšia - bloky udržiavame zoradené podľa poradia, v akom sme ich vložili do \cache. Keď vyberáme blok na odstránenie, vezmeme ten, ktorý bol pridaný najskôr.

Ak by sa táto stratégia správala tak, že vždy odstráni blok, ktorý bude potrebný v ďalšom kroku algoritmu, tak by bolo možné efektívne pracovať len s jedným blokom v \cache, napriek tomu, že by sa ich do \cache zmestilo viac. Vzhľadom na to, že analýza množstva \obliv algoritmov predpokladá istý minimálny počet blokov, ktorý sa zmestí do \cache, bol by tento predpoklad nenaplnený. To by mohlo spôsobiť, že by algoritmus vykonal viac pamäťových presunov ako táto analýza predpovedala.

Ďalším problémom je takzvaná \emph{asociatívnosť} cache -- počet pozícií v \cache, na ktoré môžeme daný blok uložiť, ktorý je z praktických dôvodov často obmedzený. Inak by bolo potrebné ukladať spolu s každým blokom jeho plnú adresu na disku, čo by redukovalo celkový počet blokov, ktoré sa do \cache zmestia. Znížením asociativity je možné ukladať iba časť adresy, pričom zvyšok je implicitne určený pozíciou v \cache. V prípade nízkej asociativity však môžu opäť nastať situácie, kedy je algoritmus schopný využiť iba malý počet blokov v \cache. V moderných systémoch (tabuľka \ref{tbl:memory-levels}) sa asociativita pohybuje okolo $8$, čo znamená, že daný blok je v \cache možné umiestniť len na $8$ z $\frac{M}{B}$ pozícií.

\subsubsection{Ideálna \cache}

Tieto problémy \obliv model obchádza predpokladom ideálnej \cache, ktorá je plne asociatívna (každý blok disku je možné uložiť v každom bloku \cache) a používa optimálnu stratégiu výmeny stránok, ktorá vždy odstráni blok, ktorý bude potrebný najneskôr. Prvý predpoklad je síce v reálnych systémoch nepraktický, no z teoretického hľadiska je v poriadku. Druhý predpoklad je nerealizovateľný, keďže by stratégia výmeny stránok musela predpovedať budúce kroky algoritmu. Nasledovné lemy však ukazujú, že aj bez týchto predpokladov na reálnom systéme s nízkou asociativitou a jednoduchou stratégiou výmeny stránok sa algoritmus zhorší len o konštantný faktor.

% TODO better translation - more exact + explicit corollaries?
% TODO change from 'lema' to 'tvrdenie' or something?
% TODO proof sketch?
\begin{lema}
Algoritmus, ktorý v ideálnej \cache veľkosti $M$ s blokmi veľkosti $B$ vykoná $T$ pamäťových operácií, vykoná najviac $2T$ pamäťových operácií v \cache veľkosti $2M$ s blokmi veľkosti $B$ pri použití stratégie \emph{LRU} alebo \emph{FIFO}. \citep[Lemma 12]{frigo1999cache}
\end{lema}

\begin{lema}
Plne asociatívna \cache veľkosti $M$ sa dá simulovať s použitím $\bigO{M}$ pamäte tak, že prístup ku každému bloku v \cache zaberie v priemernom prípade $\bigO{1}$ času. \citep[Lemma 16]{frigo1999cache}
\end{lema}

\section{Prehľad výsledkov v cache-oblivious modeli}

V tejto práci sa budeme venovať prevažne problému vyhľadávacích stromov, konkrétne B-stromom. V nasledujúcej kapitole uvedieme ich fungovanie a analýzu počtu pamäťových presunov v \extmem modeli. V tretej kapitole popíšeme vizualizácie, ktoré sme pre tieto štruktúry vytvorili. Pre širší rozhľad v tejto problematike uvádzame v tabuľke \ref{tbl:alg-overview} na strane \pageref{tbl:alg-overview} zoznam niekoľkých bežných problémov. Ku každému uvádzame ako príklad najlepšie známe \aware a \obliv algoritmy a dátové štruktúry, ktoré ten problém riešia, spolu s ich asymptotickou analýzou.

{\renewcommand{\arraystretch}{1.5}
\begin{sidewaystable}[p]
    \centering
    \caption{Prehľad výsledkov \aware a \obliv algoritmov a dátových štruktúr pre rôzne problémy. Odhady udávajú počet pamäťových operácií, pričom $M$ a $B$ sú parametre \cache.}
    \label{tbl:alg-overview}
    \begin{threeparttable}
        \begin{tabular}{r @{\hskip 1.5cm} l@{\hskip 1cm}l@{\hskip 1cm}l @{\hskip 1cm} l}
            \toprule
%            \multicolumn{1}{c}{\textbf{Úroveň}} & \multicolumn{1}{c}{\textbf{Veľkosť}} & \multicolumn{2}{c}{\textbf{Odozva}} & \multicolumn{1}{c}{\textbf{Asociativita}} & \multicolumn{1}{c}{\textbf{Veľkosť bloku}}\\ \toprule
%            \multicolumn{1}{c}{\textbf{Názov}} & \multicolumn{3}{c}{\textbf{Operácie}} & \\ \toprule
            \textbf{Vyhľadávacie stromy} & \textbf{Vyhľadávanie} & \textbf{Vkľadanie} & \textbf{Prechod}\tnote{2} & \\ \toprule
            \aware B-strom & $\bigO{\log_B{N}}$ & $\bigO{\log_B{N}}$ & $\bigO{\frac{K}{B}}$ & \citep{bayerbtree,vitteralgds} \\ 
            \obliv B-strom & $\bigO{\log_B{N}}$ & $\bigO{\log_B{N}+\frac{\log^2{N}}{B}}$\tnote{1} & $\bigO{\frac{K}{B}}$ & \citep{btreeshort,btreefull,bender2002}, časť \ref{sec:dynamic-obliv} \\
            \obliv B-strom & $\bigO{\log_B{N}}$ & $\bigO{\log_B{N}}$\tnote{1} & & \citep{btreeshort,btreefull,bender2002}, časť \ref{sec:dynamic-obliv-improved} \\ 
            \toprule
            \textbf{Usporiadaná postupnosť} & \textbf{Vkľadanie} & \textbf{Prechod}\tnote{2} & & \\ \toprule
            \aware spájaný zoznam & $\bigO{1}$\tnote{1} & $\bigO{\frac{K}{B}}$ & & \citep{demaineoverview,pagh2003basic} \\ 
            \obliv usporiadané pole & $\bigO{\frac{\log^2{N}}{B}}$\tnote{1} & $\bigO{\frac{K}{B}}$ & & \citep{btreeshort}, časť \ref{sec:orderedfile} \\
            \obliv usporiadané pole & $\bigO{\frac{\log^2{N}}{B}}$ & $\bigO{\frac{K}{B}}$ & & \citep{bender2002ofworstcase} \\
            \toprule
            \textbf{Triedenie} & & & & \\ \toprule    
            \aware mergesort & $\bigO{\frac{N}{B}\log_{M/B}{\frac{N}{B}}}$ & & & \citep{demaineoverview,vitteralgds} \\
            \obliv funnelsort & $\bigO{\frac{N}{B}\log_{M/B}{\frac{N}{B}}}$ & & & \citep{arge2005cache, demaineoverview} \\
            \toprule
            \textbf{Prioritné fronty} & \textbf{Vkľadanie} & \textbf{Výber minima} & & \\ \toprule
            \aware buffer tree & $\bigO{\frac{1}{B}\log_{M/B}{\frac{N}{B}}}$\tnote{1} & $\bigO{\frac{1}{B}\log_{M/B}{\frac{N}{B}}}$\tnote{1} & & \citep{arge1995buffer} \\
            \obliv prioritná fronta & $\bigO{\frac{1}{B}\log_{M/B}{\frac{N}{B}}}$\tnote{1} & $\bigO{\frac{1}{B}\log_{M/B}{\frac{N}{B}}}$\tnote{1} & & \cite{arge2005cache, demaineoverview} \\
            \bottomrule
        \end{tabular}
        \begin{tablenotes}
            \item[1] Amortizovaná zložitosť
            \item[2] Prechod $K$ po sebe idúcich prvkov
        \end{tablenotes}
    \end{threeparttable}
\end{sidewaystable}
}


%\begin{sidewaystable}[p]
%    \centering
%    \caption{todo}
%    \label{tbl:alg-overview}
%    \begin{threeparttable}
%        \begin{tabular}{r c*6 l}
%            \toprule
%%            \multicolumn{1}{c}{\textbf{Úroveň}} & \multicolumn{1}{c}{\textbf{Veľkosť}} & \multicolumn{2}{c}{\textbf{Odozva}} & \multicolumn{1}{c}{\textbf{Asociativita}} & \multicolumn{1}{c}{\textbf{Veľkosť bloku}}\\ \toprule
%%            \multicolumn{1}{c}{\textbf{Názov}} & \multicolumn{3}{c}{\textbf{Operácie}} & \\ \toprule
%            \textbf{Vyhľadávacie stromy} & \multicolumn{2}{c}{\textbf{Vyhľadávanie}} & \multicolumn{2}{c}{\textbf{Vkľadanie}} & \multicolumn{2}{c}{\textbf{Prechod}} & \\ \toprule
%            \aware B-strom & \multicolumn{2}{l}{$\bigO{\log_B{N}}$} & \multicolumn{2}{l}{$\bigO{\log_B{N}}$} & \multicolumn{2}{l}{$\bigO{\frac{K}{B}}$} & ref \\ 
%            \obliv B-strom & \multicolumn{2}{l}{$\bigO{\log_B{N}}$} & \multicolumn{2}{l}{$\bigO{\log_B{N}+\frac{\log^2{N}}{B}}$} & \multicolumn{2}{l}{$\bigO{\frac{K}{B}}$} & ref \\
%            \obliv B-strom & \multicolumn{2}{l}{$\bigO{\log_B{N}}$} & \multicolumn{2}{l}{$\bigO{\log_B{N}}$} & & & ref \\ 
%            \toprule
%            \textbf{Spájaný zoznam} & \multicolumn{3}{c}{\textbf{Vkľadanie}} & \multicolumn{3}{c}{\textbf{Prechod}} & \\ \toprule
%            \aware zoznam & \multicolumn{3}{c}{B} & \multicolumn{3}{c}{C} & ref \\ 
%            \obliv usporiadané pole & \multicolumn{3}{c}{B} & \multicolumn{3}{c}{C} & ref \\
%            \bottomrule
%        \end{tabular}
%        \begin{tablenotes}
%            \item[1] todo Amort
%        \end{tablenotes}
%    \end{threeparttable}
%\end{sidewaystable}