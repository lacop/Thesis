\section{Úvod}

Pri tvorbe efektívnych algoritmov sa najčastejšie zaujímame o časovú zložitosť a prípadne aj pamäťovú zložitosť. Dôležitým faktorom však môže byť aj množstvo pamäťových operácií. Tie majú priamy vplyv na výslednú časovú zložitosť avšak pri asymptotickej analýze sa často považujú za operáciu vykonateľnú v konštantnom čase. V prípadoch keď pracujeme s veľkým objemom dát, ktoré sú uložené na médiu s veľkou kapacitou ale nízkou prístupovou rýchlosťou, môže byť výhodné minimalizovať množstvo zápisov a čítaní z tohto média.

\

Klasickým prístupom v týchto prípadoch boli takzvané \aware algoritmy, ktoré potrebujú poznať presné parametre pamäťového systému pre dosiahnutie požadovanej efektivity. V tejto práci sa budeme venovať \obliv algoritmom, ktoré sú asymptoticky rovnako efektívne ako \aware, avšak bez nutnosti poznať parametre pamäte.

\

Okrem samozrejmej výhody, kedy nám stačí jedna univerzálna implementácia algoritmu pre ľubovolné množstvo rôznych systémov, prinášajú tieto algoritmy aj iné zlepšenia. V takmer každom systéme je pamäť hierarchická, zložená z viacerých úrovní, pričom každá je väčšia ale pomalšia ako predošlá. Pri \aware by pre optimálne využitie každej z úrovní pamäte bolo potrebné poznať parametre každej úrovne, a rekurzívne vnoriť do seba mnoho inštancií, každú optimalizovanú pre jednu úroveň. No v \obliv nám stačí jedna inštancia - keďže nepozná parametre žiadnej úrovne, bude rovnako dobre fungovať na každej z nich.

\

Súčasťou práce je vysvetliť problematiku \obliv pamäťového modelu, uviesť prehlad algoritmov a dátových štruktúr a vytvoriť vizualizácie vybraných z nich. Vizualizácie sú implementované ako súčasť programu {\em Alg-Vis}. Cieľom týchto vizualizácií je umožniť užívateľom experimentovať s implementáciou týchto algoritmov a dátových štruktúr, sledovať ich prácu krok po kroku a uľahčiť ich porozumeniu.