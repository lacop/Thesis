\unnumberedchapter{Úvod}
% TODO epigraph?

Pri tvorbe efektívnych algoritmov sa najčastejšie zaujímame o časovú zložitosť a prípadne aj pamäťovú zložitosť. Dôležitým faktorom však môže byť aj množstvo pamäťových operácií. Tie majú priamy vplyv na výslednú časovú zložitosť, avšak pri asymptotickej analýze sa často považujú za operáciu vykonateľnú v konštantnom čase. V prípadoch keď pracujeme s veľkým objemom dát, ktoré sú uložené na médiu s veľkou kapacitou ale nízkou prístupovou rýchlosťou, však tento predpoklad prestáva byť platný a preto môže byť výhodné minimalizovať množstvo zápisov a čítaní z tohto média. 

V skutočnosti však nemusí ísť priamo o veľmi veľký objemy dát a pomalé médiá -- rozdiel v prístupových rýchlostiach medzi vyrovnávacou pamäťou na procesore a hlavnou operačnou pamäťou je dostatočne veľký na to, aby sa nám v praxi oplatilo minimalizovať počet presunov aj medzi nimi.

Klasickým prístupom v týchto prípadoch sú takzvané \aware algoritmy, ktoré potrebujú poznať presné parametre pamäťového systému pre dosiahnutie požadovanej efektivity. Následkom toho môže byť viazanosť na konkrétny systém či náročnosť a komplikovanosť implementácie. V tejto práci sa budeme venovať \obliv algoritmom a dátovým štruktúram, ktoré tieto parametre nepoznajú, no napriek tomu sú v istých prípadoch asymptoticky rovnako efektívne ako ich \aware ekvivalenty.

Okrem samozrejmej výhody, kedy nám stačí jedna univerzálna implementácia algoritmu pre ľubovolné množstvo rôznych systémov, prinášajú tieto algoritmy aj iné zlepšenia. V takmer každom systéme je pamäť hierarchická -- zložená z viacerých úrovní, pričom každá je väčšia ale pomalšia ako tá predošlá. Pri \aware by pre optimálne využitie každej z úrovní pamäte bolo potrebné poznať parametre každej úrovne, a rekurzívne vnoriť do seba mnoho inštancií, každú optimalizovanú pre jednu úroveň. No v prípade \obliv algoritmov a dátových štruktúr nám stačí jedna inštancia - keďže nevyžaduje poznať parametre žiadnej úrovne, bude rovnako efektívna bez ohľadu na ich hodnoty a teda rovnako efektívna na každej úrovni súčasne.

V tejto práci vysvetľujeme problematiku analýzy počtu pamäťových operácií, popisujeme \obliv pamäťový model, porovnávame ho s \aware modelom a uvádzame prehľad vybraných algoritmov a dátových štruktúr. Pre popísané štruktúry uvádzame analýzu ich správania v \obliv modeli.

Hlavným výsledkom práce je implementácia vizualizácií týchto dátových štruktúr. Vizualizácie sú implementované ako rozšírenie programu {\em Gnarley trees}, ktorý vznikol ako bakalárska práca Jakuba Kováča a neskôr bol rozšírený o ďalšiu funkcionalitu v ročníkových projektoch a bakalárskych prácach. Tento program sme rozšírili o podporu pre simuláciu vyrovnávacej pamäte a pridali sme vizualizácie niekoľkých \obliv dátových štruktúr. Cieľom je umožniť užívateľom experimentovať s týmito štruktúrami, sledovať ich správanie krok po kroku. Tieto kroky sú podrobne vysvetlené čo uľahčuje porozumeniu ich fungovania. 